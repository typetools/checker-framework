\htmlhr
\chapter{Tainting checker\label{tainting-checker}}

The tainting checker prevents certain kinds of trust errors.
A \emph{tainted}, or untrusted, value is one that comes from an arbitrary,
possibly malicious source, such as user input or unvalidated data.
Using a tainted value in certain parts of your application can compromise
its integrity, causing it to crash, corrupt data, leak private data, etc.

% Ought to have many more examples

For example, a user-supplied pointer, handle, or map key should be
validated before being dereferenced.
As another example, user-supplied strings should not be concatenated into a
SQL query, program be subject to a 
\ahref{http://en.wikipedia.org/wiki/Sql_injection}{SQL injection} attack.
A location in your program where malicious data could do damage is
called a \emph{sensitive sink}.

A program can ``sanitize'' or ``untaint'' a value in two ways:  by checking
that it is innocuous/legal (e.g., it contains no characters that can be
interpreted as SQL commands when pasted into a string context), or by
transforming the value to be legal (e.g., quoting all the characters that
can be interpreted as SQL commands).  A correct program must use one of
these two techniques so that tainted values never flow to a sensitive sink.
The Tainting Checker ensures that your program does so.

If the Tainting Checker issues no warning for a given program, then no
tainted value ever flows to a sensitive sink.  However, your program is not
necessarily free from all trust errors.  As a simple example, you might
have forgotten to annotate a sensitive sink as requiring an untainted type.


\section{Tainting annotations\label{tainting-annotations}}

% TODO: add both qualifiers explicitly, and then describe their relationship.

The Tainting type system uses one annotation:
\code{@\refclass{tainting/quals}{Untainted}}.  The annotation indicates
a type that includes only untainted, trusted values.

Any type not marked as \code{Untainted} is treated as tainted, or untrusted.


\section{Writing \code{@Untainted} annotations\label{writing-untainted}}

In order to use the tainting checker, you must annotate your code with
\code{@\refclass{tainting/quals}{Untainted}} type annotation, to mark
operations that require trusted operations.

It is helpful to start annotating the secure kernel boundary entry
points.  To secure against SQL injection attacks, it is useful to
start annotating the \sunjavadoc{java/sql/Statement.html}{Statement}
class; the execute operations may only operate on untainted queries
(Chapter~\ref{annotating-libraries} describes how you can annotate
external libraries)

\begin{Verbatim}
  public boolean execute(@Untainted String sql) throws SQLException;
  public boolean executeUpdate(@Untainted String sql) throws SQLException; 
\end{Verbatim}

The Tainting checker, in turn, will verify that no untainted value may flow into
any of these methods.

% LocalWords:  quals untaint
