\htmlhr
\chapter{Units Checker\label{units-checker}}

For many applications, it is important to use the correct units of
measurement for primitive types; for example, an application that
deals with speed computations needs to ensure consistent usage of
distance and time values and that speeds are calculated by divisions
of distances and times.

The \emph{Units Checker} ensures consistent usage of units.
For example, consider the following code:

\begin{alltt}
@m int meters = 5 * UnitsTools.m;
@s int secs = 2 * UnitsTools.s;
@mPERs int speed = meters / secs;
\end{alltt}

The variables \code{meters} and \code{secs} are guaranteed to contain
only values with meters and seconds as units of measurement.
Utility class \code{UnitsTools} provides constants with which
unqualified integer are multiplied to get values of the corresponding unit.
The assignment of an unqualified value to \code{meters}, as in
\code{meters = 99}, will be flagged as an error by the Units Checker.

The division \code{meters/secs} takes the types of the two operands
into account and determines that the result is of type
meters per second, signified by the \code{@mPERs} qualifier.
We provide an extensible framework to define the result of operations
on units.


\section{Units annotations\label{units-annotations}}

The checker currently supports two kinds of units annotations:
kind annotations (\code{@Length}, \code{@Mass}, \dots) and
the SI units (\code{@m}, \code{@kg}, \dots).


Kind annotations can be used to declare what the expected unit of
measurement is, without fixing the particular unit used.
For example, one could write a method taking a \code{@Length} value,
without specifying whether it will take meters or kilometers.
We currently provide the following kind annotations:

\begin{description}
\item[\code{@Area}]

\item[\code{@Current}]

\item[\code{@Length}]

\item[\code{@Luminance}]

\item[\code{@Mass}]

\item[\code{@Speed}]

\item[\code{@Substance}]

\item[\code{@Temperature}]

\item[\code{@Time}]
\end{description}

\medskip


For each kind of unit, we provide the corresponding SI unit of
measurement.
We currently provide:

\begin{enumerate}
\item For \code{@Area}:
  the derived units
  square millimeters \code{@mm2},
  square meters \code{@m2}, and
  square kilometers \code{@km2}

\item For \code{@Current}:
  Ampere \code{@A}

\item For \code{@Length}:
  Meters \code{@m}
  and the derived units
  millimeters \code{@mm} and
  kilometers \code{@km}

\item For \code{@Luminance.java}:
  Candela \code{@cd}

\item For \code{@Mass.java}:
  kilograms \code{@kg}
  and the derived unit
  grams \code{@g}

\item For \code{@Speed.java}:
  meters per second \code{@mPERs} and
  kilometers per hour \code{@kmPERh}

\item For \code{@Substance.java}:
  Mole \code{@mol}

\item For \code{@Temperature.java}:
  Kelvin \code{@K}
  and the derived unit
  Celsius \code{@C}

\item For \code{@Time.java}:
  seconds \code{@s}
  and the derived units
  minutes \code{@min} and
  hours \code{@h}
\end{enumerate}


It is easy to specify SI unit prefixes.
Enumeration \code{Prefix} specifies all possible SI prefixes.
The basic SI units
(\code{@s}, \code{@m}, \code{@g}, \code{@A}, \code{@K},
 \code{@mol}, \code{@cd})
take an optional \code{Prefix} enum as argument.
For example, to use nano seconds as unit, one could use
\code{@s\ttlcb{}Prefix.nano\ttrcb{}} as unit type.

Class \code{UnitsTools} contains a constant for each SI unit.
To create a value of the particular unit, multiply an unqualified
value with one of these constants.
By using static imports, this allows very natural notation; for
example, after statically importing \code{UnitsTools.m},
the expression \code{5 * m} represents five meters.
As all these unit constants are public, static, and final with value
one, the compiler will optimize away these multiplications.


\section{Extending the Units Checker}

It is easy to extend and adapt the Units Checker to the particular
needs of a project.


To use nano seconds as unit, one could use
\code{@s\ttlcb{}Prefix.nano\ttrcb{}} as unit type.
However, there is a nicer way to introduce nano seconds: add an
additional qualifier.
This can be done purely declaratively:

\begin{alltt}
@Documented
@Retention(RetentionPolicy.RUNTIME)
@TypeQualifier
@SubtypeOf( \ttlcb{} Time.class \ttrcb{} )
@UnitsMultiple(quantity=s.class, prefix=Prefix.nano)
public @interface ns \ttlcb{}\ttrcb{}
\end{alltt}

The \code{@SubtypeOf} annotation specifies that this annotation
introduces an additional unit of time.
The \code{@UnitsMultiple} annotation specifies that this annotation
should be a nano multiple of the basic unit \code{@s}.
With this additional annotation, \code{@ns} and
\code{@s\ttlcb{}Prefix.nano\ttrcb{}}
behave equivalently and interchangeably.


It is also easy to introduce additional kinds of units by copying and
adapting one of the existing kinds.

To take full advantage of the additional unit qualifier, one needs to
do two additional steps:
(1)~provide a constant to convert from unqualified types to types that use
the new unit and
(2)~put the new unit in relation to existing units.

It is very easy to provide a constant to convert
unqualified types to types that use the new units,
simply by suppressing the warnings in these few locations;
see class \code{UnitsTools} for examples.

To put different units into a relationship, one provides an
implementation of the \code{UnitsRelations} interface as a
meta-annotation to one of the units.

See demonstration \code{demos/units-extension/} for an example
extension.



\section{What the Units Checker checks\label{units-checks}}

The Units Checker ensures that unrelated types are not mixed. 

All types with a particular unit annotation are
disjoint from all unannotated types, from all types with a different unit
annotation, and from all types with the same unit annotation but a
different prefix.

Subtyping between the units and the unit kinds is taken into account,
as is the \code{@UnitsMultiple} meta-annotation.

Multiplying a scalar with a unit type results in the same unit type.

The division of a unit type by the same unit type
results in the unqualified type.

Multiplying or dividing different unit types, for which no unit
relation is known to the system, will result in a \code{MixedUnits}
type, which is separate from all other units.
If you encounter a \code{MixedUnits} annotation in an error message,
ensure that your operations are performed on correct units or refine
your \code{UnitsRelations} implementation.



\section{Running the Units Checker\label{units-running}}

The Units Checker can be invoked by running the following commands.

\begin{itemize}
\item
If your code uses only the SI units that are provided by the
framework, simply invoke the checker:

\begin{Verbatim}
  javac -processor checkers.units.UnitsChecker MyFile.java ...
\end{Verbatim}

\item 
If you define your own units, provide the name of the annotations using the
\code{-Aunits} option:

\begin{alltt}
  javac -processor checkers.units.UnitsChecker
        \textit{-Aunits=myproject.quals.MyUnit} MyFile.java ...
\end{alltt}
\end{itemize}



\section{Suppressing warnings\label{units-suppressing}}

One example of when you need to suppress warnings is when you
initialize a variable with a unit type by a literal value.
To remove this warning message, it is best to introduce a
constant that represents the unit and to
add a \code{@SuppressWarnings}
annotation to that constant.
For examples, see class \code{UnitsTools}.


\section{References\label{units-references}}

\begin{itemize}
\item The GNU Units tool provides a comprehensive list of units:\\
  \ahrefurl{http://www.gnu.org/software/units/}

\item The F\# units of measurement system provided some ideas for
simplifications in our checker:\\
  \ahrefurl{http://en.wikibooks.org/wiki/F\_Sharp\_Programming/Units\_of\_Measure}

\end{itemize}
