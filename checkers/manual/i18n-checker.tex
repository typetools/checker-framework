\htmlhr
\chapter{Internationalization checker\label{i18n-checker}}

The Internationalization Checker implements a type-checker for detecting
internationalization type system.  Internationalization is the process
of making the software adopted to different languages and locales.  The
Checker ensures that the project is localizable and that any user visible
output has been localized into her language and locale.

An Internationalization checker checks for violation (and verifies the
absence) of two properties:

\begin{enumerate}

\item only proper localized text gets emitted to the user.  This property
catches errors related to outputting \code{String} literals for example.

\item only proper localizing keys are used.  This property catches errors
related to misspellings of the localization keys, or using a localization
key not found in the localization files.

\end{enumerate}

Java provides a convenient mechanism for localization through
\sunjavadoc{java/util/ResourceBundle.html}{ResourceBundle}.  The ``Java
Internationalization: Localization with ResourceBundles''
 (\myurl{http://java.sun.com/developer/technicalArticles/Intl/ResourceBundles/}).

\section{Internationalization annotations}

The Internationalization Checker supports two annotations:

\begin{enumerate}
\item \code{@\refclass{i18n/quals}{Localized}}: indicates that the qualified
\code{String} is message that has been localized and/or formatted with
respect to the used locale.

\item \code{@\refclass{i18n/quals}{LocalizableKey}}: indicates that the
qualified \code{String} or \code{Object} is a valid key found in the
localization resource files.
\end{enumerate}

\section{Running the Checker and options}

The Internationalization checker can be invoked by running the following
command:

\begin{Verbatim}
	javac -processor checkers.i18n.I18nChecker -Abundlename=MyResource <java files> ...
\end{Verbatim}

The Checker accepts two parameters (you should only specify one):

\begin{enumerate}

\item \code{-Abundlename=<resource\_name>}: where \code{resource\_name} is the
name of the resource to be used with
\sunjavadoc{java/util/ResourceBundle.html#getBundle(java.lang.String, java.util.Locale, java.lang.ClassLoader)}{ResourceBundle.getBundle()}.
The checker uses the default \code{Locale} and \code{ClassLoader} in the
compilation system.

\item \code{-Apropfile=<prop\_file>}: where \code{prop\_file} is the name of
the message properties file that maps localization keys to localized message.

\end{enumerate}
