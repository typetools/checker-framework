\htmlhr
\chapter{Fake Enum checker\label{fenum-checker}}

Since version 5 Java provides support for 
\ahref{http://java.sun.com/docs/books/jls/third\_edition/html/classes.html\#8.9}{enums}.
If possible, one should use the language provided functionality and use enums.
However, sometimes you cannot use enums.
You might have a public API that uses \code{int} constants and cannot break your existing clients.
Or in environments with limited resources, 
\ahref{http://developer.android.com/guide/practices/design/performance.html\#avoid\_enums}{avoiding enums}
can improve performance.

Using the Fake Enum checker, fenum for short, we can improve the type safety
of code that uses the fake, or straw-man, enum pattern.
A fenum introduces a new type that is distinct from all values of the base type.
Fenums can be introduced for primitive types as well as for reference types.


\section{Fake enum annotations}

The checker supports two ways to introduce a new fenum:

\begin{enumerate}
\item Introduce your own specialized fenum annotation with code like this:

\begin{Verbatim}
package myproject.quals;

import java.lang.annotation.*;
import checkers.quals.SubtypeOf;
import checkers.quals.TypeQualifier;

@Documented
@Retention(RetentionPolicy.RUNTIME)
@TypeQualifier
@SubtypeOf( { FenumTop.class } )
public @interface MyFenum {}
\end{Verbatim}

\item Use the provided \code{@\refclass{fenum/quals}{Fenum}} annotation, that takes a
\code{String} argument to distinguish different fenums.
For example, \code{@Fenum("A")} and \code{@Fenum("B")} are two distinct fenums.
\end{enumerate}


\section{Running the Fake Enum checker}

The fenum checker can be invoked by running the following
command:

\begin{Verbatim}
  javac -processor checkers.fenum.FenumChecker -Aqual=myproject.quals.MyFenum MyFile.java ...
\end{Verbatim}

If you use the \code{@\refclass{fenum/quals}{Fenum}} annotation, you do not need a checker option.
If you define your own annotation, provide the name of the annotation using the \code{-Aqual} option.


\section{Expected Warnings}

As all fenum types are distinct from unqualified values, you will get a warning
message from the checker when you initialize the fenum values.
To remove this warning message, add the corresponding \code{@SuppressWarnings} to either
the field or class declaration, for example:

\begin{Verbatim}
@SuppressWarnings("fenum:assignment.type.incompatible")
class MyConsts {
  public static final @Fenum("A") int ACONST1 = 1;
  public static final @Fenum("A") int ACONST2 = 2;  
}
\end{Verbatim}

The fenum checker also forbids method calls on fenums and comparisons between fenums and
unqualified variables. In case this is too restrictive, add the corresponding
\code{@SuppressWarnings}.  


\section{Example}

The following example introduces two fenums in class TestStatic
and then performs a few typical operations.

\begin{Verbatim}
@SuppressWarnings("fenum:assignment.type.incompatible")
public class TestStatic {
  public static final @Fenum("A") int ACONST1 = 1;
  public static final @Fenum("A") int ACONST2 = 2;
  public static final @Fenum("A") int ACONST3 = 3;

  public static final @Fenum("B") int BCONST1 = 4;
  public static final @Fenum("B") int BCONST2 = 5;
  public static final @Fenum("B") int BCONST3 = 6;
}

class FenumUser {
  @Fenum("A") int state1 = TestStatic.ACONST1;
  // Incompatible fenums forbidden!
  @Fenum("B") int state2 = TestStatic.ACONST1;

  void bar(@Fenum("A") int p) {}
	
  void foo() {
    // Direct use of value forbidden!
    state1 = 4;
    // Incompatible fenums forbidden!
    state1 = TestStatic.BCONST1;
    // ok
    state1 = TestStatic.ACONST2;
		
    // Direct use of value forbidden!
    bar(5);
    // Incompatible fenums forbidden!
    bar(TestStatic.BCONST1);
    // ok
    bar(TestStatic.ACONST1);
  }
 }
\end{Verbatim}

