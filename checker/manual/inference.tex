\htmlhr
\chapter{Type inference\label{type-inference}}

There are two different tasks that are commonly called ``type inference'':
\begin{enumerate}
\item
  Type inference during type-checking:
  The type-checker fills in an appropriate type where the programmer didn't
  write one, but does not change the source code.
  See Section~\ref{type-inference-refinement}.
\item
  Type inference to annotate a program:
  As a separate step before type-checking, a type inference tool
  inserts type qualifiers into the source code.
  See Section~\ref{type-inference-to-annotate}.
\end{enumerate}

Each variety has its own advantages, discussed below.
Advantages of \emph{all} varieties of type inference include:
\begin{itemize}
\item
  Less work for the programmer.
\item
  The tool chooses the most general type, whereas a programmer might
  accidentally write a more-specific, less-generally-useful annotation.
\end{itemize}


\section{Local type inference during type-checking\label{type-inference-refinement}}

During type-checking, if certain variables have no type qualifier, the
type-checker determines whether there is some type qualifier that would
permit the program to type-check.  If so, the type-checker uses that type
qualifier, but does not change the source code.  Each time the
type-checker runs, it re-infers the type qualifier for that variable.  If
no type qualifier exists that permits the program to type-check, the
type-checker issues a warning.

Local type inference is built into the Checker Framework.
Every checker automatically uses it.  As a result, a programmer typically
does not have to write any qualifiers inside the body of a method.
However, it primarily
works within a method, not across method boundaries.
The source code must already contain annotations for method
signatures (arguments and return values) and fields.

Advantages of this variety of type inference include:
\begin{itemize}
\item
  If the type qualifier is obvious to the programmer, then omitting it
  can reduce annotation clutter in the program.
\item
  If the code changes, then there is no old annotation that
  might need to be updated.
\item
  Within-method type inference occurs automatically.
  The programmer doesn't have to do anything to take advantage of it.
\end{itemize}

For more details about local type inference during type-checking, also
known as ``flow-sensitive local type refinement'', see
Section~\ref{type-refinement}.


\section{Type inference to annotate a program\label{type-inference-to-annotate}}

As a separate step before type-checking, a type inference tool takes the
program as input, and outputs a set of type qualifiers that would
make the program type-check.
These qualifiers are inserted into the source code or the
class file.  They can be viewed and adjusted by the programmer, and can
be used by tools such as the type-checker.

Advantages of this variety of type inference include:
\begin{itemize}
\item
  The inference may be more precise by taking account of the entire program
  rather than just reasoning one method at a time.
\item
  The program source code contains documentation in the form of type
  qualifiers, which can aid programmer understanding and may make
  type-checking warnings more comprehensible.
\end{itemize}


\subsection{Type inference tools\label{type-inference-tools}}

This section lists tools that take a program and output a set of
annotations for it.


\subsubsection{Type inference for specific type systems\label{type-inference-tools-specialized}}

This section lists type inference tools that work for only a single type system.
Also see Section~\ref{type-inference-tools-general}, for type inference
tools that work for any type system.

Section~\ref{nullness-inference} lists several tools that infer
annotations for the Nullness Checker.

Section~\ref{javari-inference} lists a tool that infers
annotations for the Javari Checker, which detects mutation errors.

If you run the Checker Framework with the \<-AsuggestPureMethods>
command-line option, it will suggest methods that can be marked as
\<@SideEffectFree>, \<@Deterministic>, or \<@Pure>; see
Section~\ref{type-refinement-purity}.


\subsubsection{Type inference for any type system\label{type-inference-tools-general}}

This section lists type inference tools that work for any type system.
Also see Section~\ref{type-inference-tools-specialized}, for type inference
tools that work for only a single type system.

By supplying the \<-AinferSignatures> command-line option,
any type-checker can infer annotations. See Section~\ref{signature-inference}.

\href{https://github.com/reprogrammer/cascade/}{Cascade}~\cite{VakilianPEJ2014}
is an Eclipse plugin that implements interactive type qualifier inference.
Cascade is interactive rather than fully-automated:  it makes it easier for
a developer to insert annotations.
Cascade starts with an unannotated program and runs a type-checker.  For each
warning it suggests multiple fixes, the developer chooses a fix, and
Cascade applies it.  Cascade works with any checker built on the Checker
Framework.
You can find installation instructions and a video tutorial at \url{https://github.com/reprogrammer/cascade}.


\subsection{Signature inference (interprocedural type refinement)\label{signature-inference}}

Elsewhere in this manual, type refinement (Section~\ref{type-refinement})
has been described as working only within a method.  In fact, it can also
infer types for fields, method parameters, and method return types, and it
can insert these inferred types into your program.

To do so, run the script
\<checker/bin/infer-and-annotate.sh>.
Its command-line arguments are:

\begin{enumerate}
\item Processor's name.
\item Target program's classpath.  This argument is required; pass "" if it
  is empty.
\item Extra processor arguments which will be passed to the checker, if any.
  You may supply any number of such arguments, or none.  Each such argument
  must start with a hyphen.
\item Optional: List of paths to \<.jaif> files used as input in the inference
    process.
\item List of paths to \<.java> files in the program.
\end{enumerate}

For example, to add annotations to the \<plume-lib> project:
\begin{Verbatim}
git clone https://github.com/mernst/plume-lib.git
cd plume-lib
$CHECKERFRAMEWORK/checker/bin/infer-and-annotate.sh \
    "LockChecker,NullnessChecker" java/plume.jar    \
    -AprintErrorStack  \
    `find java/src/plume/ -name "*.java"`
# View the results
git diff    
\end{Verbatim}

You may need to wait a few minutes for the command to complete,
and it will produce voluminous output that you can ignore.

It is recommended that you run \<infer-and-annotate.sh> on a copy of your
code, so that you can see what changes it made and so that it does not
change your only copy.  One way to do this is to work in a clone of your
repository that has no uncommitted changes.

As with any type inference tool, it is a good idea to double-check the
results.
For example, suppose that your program currently calls
method \<m1> with non-null
arguments.  This tool will infer that \<m1>'s parameter has
\<@NonNull> type.  If you had intended the method to be able to
take \<null> as an argument and you later add such a call, the type-checker
will issue a warning because the automatically-inserted \<@NonNull>
annotation is inconsistent with the new call.
As another example, suppose you intend method \<m2> to be called with
non-null arguments, but your program contains an error and one of the calls
to \<m2> passes \<null> as the argument.  Then the tool will infer that
\<m2>'s parameter has \<@Nullable> type.

Another way of explaining the issue is that the tool determines a method parameter's type
annotation based on what arguments are passed to the method but not on how the
parameter is used within the method body.  If the program contains
no errors, the tool's annotations are legal for the program as
currently written but may not be as general as possible and may not
accommodate future program changes.  If the program contains errors, the
tool's annotations may reflect those errors.

The tool only infers annotations for type uses that have no user-written
annotation (for the given type system).

\subsubsection{How signature inference works\label{how-signature-inference-works}}

This section explains how the \<infer-and-annotate.sh> script works.  If you
merely want to run the script and you are not encountering trouble, you can
skip this section.

If you
supply to javac the command-line option \code{-AinferSignatures}, then the
checker outputs \<.jaif> files with refined types for fields and method signatures.
The output .jaif files are located in the folder \code{build/signature-inference},
relative to where you executed the javac command.

You can use the Annotation File Utilities
(\myurl{http://types.cs.washington.edu/annotation-file-utilities/}) to
insert these refined types in your program.  Then, the next time that you
run type-checking, there are likely to be fewer type-checking warnings.

Note that a three-step process is required:
\begin{enumerate}
\item Run the checker with the \<-AinferSignatures> command-line option to
  produce a \<.jaif> file.  Some type-checking errors may result.
\item Insert the \<.jaif> file's annotations in the program.
\item Run the checker again.  Fewer type-checking errors may result.
\end{enumerate}
\noindent
A good approach is to repeatedly run the above process until there are no
more changes to the inference results (that is, the \<.jaif> file is
unchanged between two runs).  That is exactly what the
\<infer-and-annotate.sh> script does.

The \<infer-and-annotate.sh> script insulates you from the
clumsy multi-step process.  The multi-step process
is required because type-checking is modular:
it processes each class only once, independently.  Modularity enables you
to run type-checking on only part of your program, and
it makes type-checking fast.  However, it has some disadvantages:
\begin{itemize}
\item
  The first run of the type-checker cannot not take advantage
  of signature inference results because signature inference is only complete at the
  end of type-checking, and modular type-checking does not revisit any
  already-processed classes.
\item
  The reason that multiple executions are required is that revisiting an
  already-processed class may result in a better estimate.
\end{itemize}



%%  LocalWords:  AinferSignatures java jaif plugin classpath m2 m1 multi
%%  LocalWords:  AsuggestPureMethods
