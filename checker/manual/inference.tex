\htmlhr
\chapter{Type inference\label{type-inference}}

There are two different tasks that are commonly called ``type inference'':
\begin{enumerate}
\item
  Type inference during type-checking:
  the type-checker fills in an appropriate type where the programmer didn't
  write one, but never tells the programmer what it was.
  See Section~\ref{type-inference-refinement}.
\item
  Type inference to annotate a program:
  As a separate step before type-checking, a type inference tool
  inserts type qualifiers into the source code.
  See Section~\ref{type-inference-to-annotate}.
\end{enumerate}

Each variety has its own advantages, discussed below.
Advantages of \emph{all} varieties of inference include:
\begin{itemize}
\item
  Less work for the programmer.
\item
  The tool chooses the most general type, whereas a programmer might
  accidentally write a more specific, less generally-useful annotation.
\end{itemize}


\section{Local type inference during type-checking\label{type-inference-refinement}}

During type-checking, if certain variables have no type qualifier, the
type-checker determines whether there is some type qualifier that would
permit the program to type-check.  If so, the type-checker uses that type
qualifier, but never tells the programmer what it was.  Each time the
type-checker runs, it re-infers the type qualifier for that variable.  If
no type qualifier exists that permits the program to type-check, the
type-checker issues a warning.

This variety of type inference is built into the Checker Framework.
Every checker automatically uses it.  As a result, a programmer typically
does not have to write any qualifiers inside the body of a method.
However, it primarily
works within a method, not across method boundaries.  Method
signatures (arguments and return values) and fields must have already been
explicitly annotated.

Advantages of this variety of type inference include:
\begin{itemize}
\item
  If the type qualifier is obvious to the programmer, then omitting it
  can reduce annotation clutter in the program.
\item
  The type inference can take advantage of only the code currently being
  compiled, rather than having to be correct for all possible calls.
  Additionally, if the code changes, then there is no old annotation that
  might need to be updated.
\item
  Within-method type inference occurs automatically.
  The programmer doesn't have to do anything to take advantage of it.
\end{itemize}

For more details about local type inference during type-checking, also
known as ``flow-sensitive local type refinement'', see
Section~\ref{type-refinement}.


\section{Type inference to annotate a program\label{type-inference-to-annotate}}

As a separate step before type-checking, a type inference tool takes the
program as input, and outputs a set of type qualifiers that would
type-check.  These qualifiers are inserted into the source code or the
class file.  They can be viewed and adjusted by the programmer, and can
be used by tools such as the type-checker.

Advantages of this variety of type inference include:
\begin{itemize}
\item
  The inference may be more precise by taking account of the entire program
  rather than just reasoning one method at a time.
\item
  The program source code contains documentation in the form of type
  qualifiers, which can aid programmer understanding.
\item
  Error messages may be more comprehensible.  With type inference
  during type-checking, error messages can be obscure, because the
  compiler has already inferred (possibly incorrect) types for a number
  of variables.
\end{itemize}


\subsection{Type inference tools\label{type-inference-tools}}

This section lists tools that take a program and output a set of
annotations for it.

Every type-checker can infer annotations when you supply the
\<-AuseJaifInference> command-line option.  See Section~\ref{jaif-inference}.

Section~\ref{nullness-inference} lists several tools that infer
annotations for the Nullness Checker.

Section~\ref{javari-inference} lists a tool that infers
annotations for the Javari Checker, which detects mutation errors.

\href{https://github.com/reprogrammer/cascade/}{Cascade}~\cite{VakilianPEJ2014}
is an Eclipse plugin that implements interactive type qualifier inference.
Cascade is interactive rather than fully-automated:  it makes it easier for
a developer to insert annotations.
Cascade starts with an unannotated program and runs a type-checker.  For each
warning it suggests multiple fixes, the developer chooses a fix, and
Cascade applies it.  Cascade works with any checker built on the Checker
Framework.
You can find installation instructions and a video tutorial at \url{https://github.com/reprogrammer/cascade}.


\subsection{Interprocedural type refinement during type-checking\label{jaif-inference}}

Elsewhere in this manual, type refinement (Section~\ref{type-refinement})
has been described as working only within a method.  In fact, it can also
infer types for fields, method parameters, and method return types.  If you
supply the \code{-AuseJaifInference} command-line option to javac, then the
checker outputs \<.jaif> files with refined types for the entire program.
The output .jaif files are located in the folder \code{build/jaif-files},
relative to where you executed the javac command.

You can use the Annotation File Utilities
(\myurl{http://types.cs.washington.edu/annotation-file-utilities/}) to
insert these refined types in your program.  Then, the next time that you
run type-checking, there are likely to be fewer type-checking warnings.

Note that a three-step process is required:
\begin{enumerate}
\item Run the checker with the \<-AuseJaifInference> command-line option to
  produce a \<.jaif> file.  Some type-checking errors may result.
\item Insert the \<.jaif> file's annotations in the program.
\item Run the checker again.  Fewer type-checking errors may result.
\end{enumerate}
\noindent
A good approach is to repeatedly run the above process until there are no
more changes to the infernce results (that is, the \<.jaif> file is
unchanged between two runs), before you begin to change annotations in the
program source code.
[TO DO: Automating this process would be helpful to users.  Could that be
  done within the Checker Framework, or would it need to iteratively call
  the Checker Framework?]

This slightly clumsy process is required because type-checking is modular:
it processes each class only once, independently.  This means that it is
possible to run type-checking on only part of your program, and
type-checking is fast.
\begin{itemize}
\item
  The reason that the first run of the type-checker does not take advantage
  of the inference results is that the inference is only complete at the
  end of type-checking, and type-checking does not revisit any
  already-processed classes.
\item
  The reason that multiple executions are required is that revisiting an
  already-processed class may result in a better estimate.
\end{itemize}

As with any type inference tool, it is a good idea to double-check the
results.  In particular, this type inference may yield results that are too
specific.  For example, if a particular method is only called with non-null
arguments, then this process will infer that the method's parameter has
@NonNull type.  If the programmer had intended the method to be able to
take \<null> as an argument and later adds such a call, the type-checker
will issue a warning until you manually change the method parameter's type
annotation.  Another way of saying this is that a method parameter's type
annotation depends on what arguments are passed to it but not on how the
parameter is used within the method body, and that (if the program contains
no errors) the result is a typing that is legal for the program as
currently written but may not be as general as possible and may not
accommodate future program changes.


[TO DO:  Does .jaif file inference always result in a new type that is a
  subtype of the declared type, or might it result in a supertype?  For
  starting from an unannotated program, supertypes could be useful to infer
  @Nullable which is different from the default which is @NonNull.  Could
  this be controlled by a command-line option or two?]


\subsection{Older text}

To reduce the amount of manually added annotations required to
type-check a program, you can run the \code{infer-and-annotate} script.
The script performs a whole-program type inference using .jaif files and is
able to infer types for class fields and method return types.
The script receives as arguments:

\begin{enumerate}
\item Processor's name.
\item Target project's classpath.
\item Target project's source folder path.
\item Extra processor arguments (optional).
\end{enumerate}

The script outputs a folder named "annotated" on the project's root.
This folder contains annotated versions of files in the source folder that
was passed as an argument.

Example of usage for the Constant Value Checker (Chapter~\ref{constant-value-checker}):
\begin{Verbatim}
$CHECKERFRAMEWORK/framework/jaif-inference-scripts/infer-and-annotate.sh \
org.checkerframework.common.value.ValueChecker PROJECT_CP PROJECT_SRC EXTRA_PROCESSOR_ARGS
\end{Verbatim}
