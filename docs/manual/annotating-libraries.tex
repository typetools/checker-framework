\htmlhr
\chapter{Annotating libraries\label{annotating-libraries}}

When your code uses a library that is not currently being compiled,
the Checker Framework looks up the library's annotations in its class files.
Section~\ref{annotated-libraries-using} tells you how to use a
version of a library that contains type annotations.
If your code uses a library that does \emph{not} contain type annotations,
then the type-checker has no way to know the library's behavior.
The type-checker
makes conservative assumptions about unannotated bytecode.
% :  it assumes that every method parameter has the bottom type annotation
% and that every method return type has the top type annotation
(See
Section~\ref{defaults-classfile} for details, an example, and how to
override this conservative behavior.)
These conservative library
annotations invariably lead to checker warnings.

This chapter describes how to eliminate
the warnings by adding annotations to the library.
(Alternately, you can instead
suppress all warnings related to an unannotated library, or to part of your
codebase, by use of the
\code{-AskipUses} or \code{-AonlyUses} command-line option; see
Sections~\ref{askipuses}--\ref{askipdefs}.)


\section{Creating an annotated library\label{annotated-libraries-creating}}

This section describes how to create an annotated library --- that is, how
to add annotations to a library so that the Checker Framework can better
type-check clients of the library.

You can write annotations for a library, and make them known to a checker, in two ways.

\begin{enumerate}
\item
  Write annotations in a copy of the library's source code.

  Compile the library to create \<.class> and \<.jar> files that
  contain the annotations.

  Then, when doing pluggable \emph{type-checking},
  put those files on the classpath.
  When \emph{running} your code, you can use either version of the library:  the
  one you created or the original distributed version.

  %% There is no point to advertising this deprecated workflow.
  % You can insert annotations in the compiled \code{.class} files of the
  % library.  You would express the annotations textually, typically as an
  % annotation index file, and
  % then insert them in the library by using the Annotation File Utilities
  % (\myurl{https://checkerframework.org/annotation-file-utilities/}).
  % See the Annotation File Utilities documentation for full details.

  With this compilation approach, the syntax of the library annotations is
  validated ahead of time.  Thus, this compilation approach is less
  error-prone, and the type-checker runs faster.  You get
  correctness guarantees about the library in addition to your code.
  Section~\ref{compiling-libraries} describes how to compile a library.

\item
  Write annotations in a ``stub file'', if you do not have access to the
  source code.
  %% Leave out this complication.
  % This approach is possible with annotated source code.

  Supply the ``stub file'' textually to the Checker Framework.

  This approach does not require you to compile the library source
  code.
  A stub file is applicable to multiple versions of a library, so
  the stub file does not need to be updated when a new version of the
  library is released.

  When provided by the author of a checker, a stub file is used
  automatically, with no need for the user to supply a command-line option.
  Section~\ref{stub} describes how to create and use stub files.

\end{enumerate}


If you annotate a new library, please inform the Checker Framework
developers so that your annotated library can be distributed with the
Checker Framework.
Sharing your annotations is useful even if the library is only partially
annotated.
However, as noted in Section~\ref{get-started-with-legacy-code}, you
should annotate an entire class at a time.
You may find type inference tools (\chapterpageref{type-inference-to-annotate}) helpful
when getting started, but you should always examine their results.
%% This text does not feel so relevant.
% You can determine the correct annotations for a library either
% automatically by running an inference tool, or manually by reading the
% documentation.  Presently, automated type inference tools are available for the
% Nullness (Section~\ref{nullness-inference}) type system.


\subsection{Tips for annotating a library\label{annotating-tips}}

When you annotate a library, you should only add annotations and, when
necessary, documentation of those annotations in a Java comment (\<//>
or \</*...*/>).

Do not change the library's code, including formatting and whitespace.
Do not change publicly-visible documentation, such as Javadoc comments.
Changes like these will increase the size of the diffs
between upstream and your version.  Unnecessary diffs make it harder for
others to understand what you have done, and they make it harder to pull
changes from upstream into the annotated library.
(The only time it is acceptable to comment out existing code in the library
is if the regular Java compiler cannot compile the code; but this should be
extremely rare.)

Your annotations should re-state facts that are in the Javadoc
documentation.  Do not add requirements or guarantees beyond what the
library author has already committed to.  Sometimes the Javadoc
documentation omits some facts.  If a project's Javadoc says nothing about
nullness, then you should not assume that the specification forbids null,
nor that it permits it.  You need to get guidance from the library author.
(If the project's Javadoc does sometimes explicitly permit/forbid null,
then you can assume the opposite at places where the author omitted those
statements.)  In some cases, you can infer some facts from the
implementation, such as that null is permitted for a method's parameter or
return type if the implementation explicitly passes null to the method or
the method implementation returns null.

While annotating the library, you may discover bugs or missing/wrong
documentation.  If you have a documentation improvement or a bug fix, then
open a pullrequest against the upstream version of the library.  This will
benefit all users of the library.

If you do not annotate all the files in the library, then use
\refqualclass{framework/qual}{AnnotatedFor} to indicate what files you have
annotated.

Whenever you annotate any part of a file, fully annotate the file!  That
is, write annotations for all the methods and fields, based on their
documentation.  If you don't annotate the whole file, then other people
won't know whether a particular method is unannotated because you didn't
get to it yet or because you decided that it didn't need any annotations.

Ideally, after you annotate a file, you can type-check the file to verify
the correctness and completeness of your annotations.  An alternative is to
only annotate method signatures.  The alternative is quicker but more
error-prone, and there is no difference from the point of view of clients,
who can only see annotations on public methods and fields.  When you
compile the library, the type-checker will probably issue warnings, but if
you supply \<-Awarns>, you don't have to suppress those warnings and the
compiler will still produce \<.class> files.


\subsection{Creating an annotated JDK\label{annotating-jdk}}

When you create a new checker, you need to also supply annotations for
parts of the JDK, either as stub files or as source code that will be
compiled and its annotations inserted into the JDK\@.  This
section describes the latter approach.
It assumes that you have already followed the instructions for building the
Checker Framework from source (Section~\ref{build-source}).

If you are adding annotations to an existing annotated JDK, then you only
need to follow the last two steps.

\begin{enumerate}
\item
Get Java 8 source code (must be version 8)
\begin{enumerate}
\item Download JDK8 from Oracle
\item Open the downloaded tar (\<tar -xvzf>)
\item Unzip the contained \<src.zip> (resulting in folders: \<com/>, \<java/>, \<javax/>, \<launcher/>, \<org/>)
\end{enumerate}

\item
Set up the Checker Framework (replace \textit{mychecker} with the name of your checker)
\begin{enumerate}
\item \<mkdir \$JSR308/checker-framework/checker/jdk/\textit{mychecker}/> \\
  \<cd \$JSR308/checker-framework/checker/jdk/\textit{mychecker}/> \\
  \code{echo "include ../Makefile.jdk" > Makefile}

\item Add \textit{mychecker} to
  \<\$JSR308/checker-framework/checker/jdk/MakeFile> in the  definition of
  \<CHECKER\_DIRS> and in the definition of \<ANNOTATED\_CLASSES> (don't forget
  to add a closing parenthesis at the end of the definition of \<ANNOTATED\_CLASSES>!).

\end{enumerate}

\item
For each file you want to annotate, copy the JDK version into the
directory
\<\$JSR308/checker-framework/checker/jdk/\textit{mychecker}/src/>, using
the same directory structure as the JDK\@.

Whenever you add a file, fully annotate it, as described in
Section~\ref{annotating-tips}.

If you are only annotating fields and method signatures (but not
ensuring that method bodies type-check), then you don't need to suppress
warnings, because the JDK is built using the \<-Awarns> command-line
option.

\item
Build the annotated JDK:
	run \<ant> in \<\$JSR308/checker-framework/checker>

You may receive a compilation error because your files use a part of the
JDK that is not in the annotated JDK\@.  If the missing part is relatively
small, please add it.  If the missing part is large, then you can make
small changes to the JDK source code, such as commenting out a method body.
(Use \</*> and \<*/> on their own lines, so that the diffs are minimal.)

\end{enumerate}


\section{Compiling partially-annotated libraries\label{compiling-libraries}}

If you completely annotate a library, then you can compile it using a
pluggable type-checker, and include the resulting \<.jar> file on your
classpath.  You get a guarantee that the library contains no errors.

The rest of this section tells you how to compile a library if you
\emph{partially} annotate it:  that is, you write annotations for some of its
classes but not others.

(There is another type of partial annotation, which is when you annotate
method signatures but do not type-check the bodies.
See Section~\ref{annotating-tips}.)

When compiling a partially-annotated library, the checker needs to use normal
defaulting rules (Section~\ref{climb-to-top}) for code you have annotated and
conservative defaulting rules (Section~\ref{defaults-classfile}) for
code you have not yet annotated.
You use \<@AnnotatedFor> to indicate which classes you have annotated.



\subsection{The \<-AuseDefaultsForUncheckedCode=source,bytecode> command-line argument\label{AuseDefaultsForUncheckedCodesource}}

\begin{sloppypar}
When compiling a library that is not fully annotated, use command-line
argument \<-AuseDefaultsForUncheckedCode=source,bytecode>.  This causes
the checker to behave normally for classes with a relevant \<@AnnotatedFor>
annotation.  For classes without \<@AnnotatedFor>, the checker uses
unchecked code defaults
(see Section~\ref{defaults-classfile}) for any type use with no explicit
user-written annotation, and the checker issues no warnings.
\end{sloppypar}

The \refqualclass{framework/qual}{AnnotatedFor} annotation, written on a
class, indicates that the class has been annotated for certain type
systems.  For example, \<@AnnotatedFor(\ttlcb"nullness", "regex"\ttrcb)> means that
the programmer has written annotations for the Nullness and Regular
Expression type systems.  If one of those two type-checkers is run,
the \<-AuseDefaultsForUncheckedCode=source,bytecode> command-line argument
has no effect and this class is treated normally:
unannotated types are defaulted using normal source-code
defaults and type-checking warnings are issued.
\refqualclass{framework/qual}{AnnotatedFor}'s arguments are any string that
may be passed to the \<-processor> command-line argument:  the
fully-qualified class name for the checker, or a shorthand for built-in
checkers (see Section~\ref{shorthand-for-checkers}).
Writing \<@AnnotatedFor> on a class doesn't necessarily mean that you wrote
any annotations, but that the user examined the source code and verified
that all appropriate annotations are present.

\begin{sloppypar}
Whenever you compile a class using the Checker Framework, including when
using the \<-AuseDefaultsForUncheckedCode=source,bytecode> command-line
argument, the resulting \<.class> files are fully-annotated; each type use
in the \<.class> file has an explicit type qualifier for any checker that
is run.
\end{sloppypar}


\subsection{Workflow for creating or augmenting a partially-annotated library\label{compiling-libraries-workflow}}

This section describes the typical workflow for creating a
partially-annotated library.

\begin{enumerate}
\item See the
  \ahref{https://search.maven.org/#search\%7Cga\%7C1\%7Cg\%3A\%22org.checkerframework.annotatedlib\%22}{the
  \<org.checkerframework.annotatedlib> group in the Central Repository}
  to find out whether an annotated version of the library already exists.
%% This adds clutter to the source code, so omit the step from these instructions.
% \item Optionally, run a script that adds
%     \<\refqualclass{framework/qual}{AnnotatedFor}(\ttlcb\ttrcb)>
%     to each class.  [[TODO: say where to find this script.]]
%     This step has no semantic effect.  It only saves you the trouble of
%     typing those 17 characters later.

  If it exists, fork its repository in \url{https://github.com/typetools/},
  tweak its buildfile to run an additional checker, and add annotations for
  that checker.

  If it does not already exist, fork the project (if its license permits
  forking).  Add a note, perhaps in a README, indicating how to obtain the
  corresponding upstream version; that will enable others to see exactly
  what edits you have made.

  Adjust the library's
  build process, such as a Maven or Ant buildfile.
  \begin{sloppypar}
  \begin{enumerate}
  \item
    Every time the build system runs the compiler, it should:
    \begin{itemize}
    \item
      passes the \<-AuseDefaultsForUncheckedCode=source,bytecode>
      command-line option and
    \item
      runs every pluggable type-checker for which any
      annotations exist, using \<-processor
      \emph{TypeSystem1},\emph{TypeSystem2},\emph{TypeSystem3}>
    \end{itemize}
  \item
    When the build system creates a \<.jar> file, the resulting \<.jar>
    file includes the contents of
    \<checker-framework/checker/dist/checker-qual.jar>.
  \end{enumerate}
  \end{sloppypar}

  You are not adding new build targets, but modifying existing targets.
  The reason to run every type-checker is to verify
  the annotations you wrote, and to use appropriate defaults for all
  unannotated type uses.
  The reason to include the contents of \<checker-qual.jar> is so that
  the resulting \<.jar> file can be used whether or not the Checker Framework
  is being run.

\item Annotate some files.
%  You can determine which files need to be annotated by using the
%  \<-AprintUnannotatedMethods> command-line argument while type-checking a
%  client of the library.

  % This is a strong recomendation, but not a requirement.  @AnnotatedFor
  % can be written on a method as well.
  When you annotate a file, annotate the whole thing, not just a few of its
  methods.  Once the file is fully annotated, add an
  \<\refqualclass{framework/qual}{AnnotatedFor}(\ttlcb"\emph{checkername}"\ttrcb)>
  annotation to its class(es), or augment an existing \<@AnnotatedFor>
  annotation.

\item
  Build the library.

  Because of the changes that you made in step 1, this will run pluggable
  type-checkers.  If there are any compiler warnings, fix them and re-compile.

  Now you have a \<.jar> file that you can use while type-checking and at
  run time.

\item
  Tell other people about your work so that they can benefit from it.

  \begin{itemize}
  \item
    Please inform the Checker Framework developers
    about your new annotated library by opening an issue.
    This will let us add your annotations to a repository in
    \url{https://github.com/typetools/} and upload a compiled artifact to
    the Maven Central Repository.

  \item
    Encourage the library's maintainers to accept your annotations into its
    main version control repository.  This will make the annotations easier
    to maintain, the library will obtain the correctness guarantees of
    pluggable type-checking, and there will be no need for the Checker
    Framework to include an annotated version of the library.

    You will probably want to write the annotations in comments, so that it
    is still possible to compile the library without use of Java 8.

    If the library maintainers do not accept the annotations, then
    periodically, such as when a new version of the library is released,
    pull changes from upstream (the library's main version control system)
    into your fork, add annotations to any newly-added methods in classes
    that are annotated with \<@AnnotatedFor>, rebuild to create an updated
    \<.jar> file, and inform the Checker Framework developers by opening an
    issue or issuing a pull request.
  \end{itemize}

\end{enumerate}


\section{Using stub classes\label{stub}\label{stub-creating-and-using}}

A stub file contains ``stub classes'' that contain annotated signatures,
but no method bodies.  A
checker uses the annotated signatures at compile time, instead of or in
addition to annotations that appear in the library.

Section~\ref{annotating-jdk} explains how you should choose between
creating stub classes or creating an annotated library.
Section~\ref{stub-creating} describes how to create stub classes.
Section~\ref{stub-using} describes how to use stub classes.
These sections illustrate stub classes via the example of creating a \refqualclass{checker/interning/qual}{Interned}-annotated
version of \code{java.lang.String}.  You don't need to repeat these steps
to handle \code{java.lang.String} for the Interning Checker,
but you might do something similar for a different class and/or checker.


\subsection{Using a stub file\label{stub-using}}

The \code{-Astubs} argument causes the Checker Framework to read
annotations from annotated stub classes in preference to the unannotated
original library classes.  For example:

\begin{myxsmall}
\begin{Verbatim}
  javac -processor org.checkerframework.checker.interning.InterningChecker -Astubs=String.astub:stubs MyFile.java MyOtherFile.java ...
\end{Verbatim}
\end{myxsmall}

Each stub path entry is a file or a directory; specifying a directory is
equivalent to specifying every file in it whose name ends with
\code{.astub}.  The stub path entries are delimited by
\<File.pathSeparator> (`\<:>' for Linux and Mac, `\<;>' for Windows).

A checker automatically reads the stub file \code{jdk.astub}, unless
command-line option \<-Aignorejdkastub> is supplied.  (The checker
author should place \code{jdk.astub} in the same directory as the Checker class, i.e.,
the subclass of \code{BaseTypeVisitor}.)  Programmers should only use the
\<-Astubs> argument for additional stub files they create themselves.

If a method appears in more than one stub file (or twice in the same
stub file), then the annotations are merged. If any of the
methods have different annotations from the same hierarchy on the same type,
then the annotation from the last declaration is used.

% \textbf{The following is not implemented yet}
% A library writers should create a file \code{library.astub} on the
% classpath (in the resources directory or the binary jars).
% The Checker Framework automatically imports all the stub files named
% \code{library.astub} found in the classpath.

If both bytecode and a stub file provide information for the same
element, the stub file information is used. In particular, an
un-annotated type variable in a stub file is used instead of
annotations on a type variable in bytecode.
This feature allows stub files to change the effective annotations in
all possible situations.
Use the \<-AstubWarnIfOverwritesBytecode> command-line option to get a
warning whenever a stub file overwrites bytecode annotations.

A file being compiled takes precedence over a stub file.  If file \<A.java>
is being compiled, then any stub for class \<A> is ignored.


\subsection{Stub file format\label{stub-format}}

Every Java file is a valid stub file.  However, you can omit information
that is not relevant to pluggable type-checking; this makes the stub file
smaller and easier for people to read and write.

As an illustration, a stub file for the Interning type system
(Chapter~\ref{interning-checker}) could be:

\begin{Verbatim}
  import org.checkerframework.checker.interning.qual.Interned;
  package java.lang;
  @Interned class Class<T> { }
  class String {
    @Interned String intern();
  }
\end{Verbatim}

Note, annotations in comments are ignored.

The stub file format is allowed to differ from Java source code in the
following ways:
\begin{description}

\item{\textbf{Method bodies:}}
  The stub class does not require method bodies for classes; any method
  body may be replaced by a semicolon (\code{;}), as in an interface or
  abstract method declaration.

\item{\textbf{Method declarations:}}
  You only have to specify the methods that you need to annotate.
  Any method declaration may be omitted, in which case the checker reads
  its annotations from library's \<.class> files.  (If you are using a stub class, then
  typically the library is unannotated.)

\item{\textbf{Declaration specifiers:}}
  Declaration specifiers (e.g., \<public>, \<final>, \<volatile>)
  may be omitted.

\item{\textbf{Return types:}}
  The return type of a method does not need to match the real method.
  In particular, it is valid to use \<java.lang.Object> for every method.
  This simplifies the creation of stub files.

\item{\textbf{Import statements:}}
  Imports may appear at the beginning of the file or after any package declaration.
  The only required import statements are the ones to import type
  annotations.  Import statements for types are optional.

\item{\textbf{Multiple classes and packages:}}
  The stub file format permits having multiple classes and packages.
  The packages are separated by a package statement:
  \<package my.package;>.  Each package declaration may occur only once; in
  other words, all classes from a package must appear together.

\end{description}



\subsection{Creating a stub file\label{stub-creating}}

Stub files are generally stored together with the checker implementation,
in the same directory as the checker's \<.java> source code.


\subsubsection{If you have access to the Java source code\label{stub-creating-with-source}}

Every Java file is a stub file.  If you have access to the Java file, then
you can use the Java file as the stub file.  Just add
annotations to the signatures, leaving the method bodies unchanged.
The stub file parser silently ignores any annotations that it cannot
resolve to a type, so don't forget the \<import> statement.

Optionally (but highly recommended!), run the type-checker to verify that
your annotations are correct.  When you run the type-checker on your
annotations, there should not be any stub file that also contains
annotations for the class.  In particular, if you are type-checking the JDK
itself, then you should use the \<-Aignorejdkastub> command-line option.

This approach retains the original
documentation and source code, making it easier for a programmer to
double-check the annotations.  It also enables creation of diffs, easing
the process of upgrading when a library adds new methods.  And, the
annotations are in a format that the library maintainers can even
incorporate.

The downside of this approach is that the stub files are larger.  This can
slow down the Checker Framework, because it parses the stub files each time
it runs.
% Furthermore, a programmer must search the stub file
% for a given method rather than just skimming a few pages of method signatures.


\subsubsection{If you do not have access to the Java source code\label{stub-creating-without-source}}

If you do not have access to the library source code, then you can create a
stub file from the class file (Section~\ref{stub-creating}),
and then annotate it.  The rest of this section describes this approach.


\begin{enumerate}

\item
  Create a stub file by running the stub class generator.  (\<checker.jar> and \<javac.jar>
  must be on your classpath.)

\begin{Verbatim}
  cd nullness-stub
  java org.checkerframework.framework.stub.StubGenerator java.lang.String > String.astub
\end{Verbatim}

  Supply it with the fully-qualified name of the class for which you wish to
  generate a stub class.  The stub class generator prints the
  stub class to standard out, so you may wish to redirect its output to a
  file.

\item
  Add import statements for the annotations.  So you would need to
add the following import statement at the beginning of the file:

\begin{Verbatim}
  import org.checkerframework.checker.interning.qual.*;
\end{Verbatim}

\noindent
The stub file parser silently ignores any annotations that it cannot
resolve to a type, so don't forget the import statement.
Use the \<-AstubWarnIfNotFound> command-line option to see warnings
if an entry could not be found.

\item
  Add annotations to the stub class.  For example, you might annotate
  the \sunjavadoc{java/lang/String.html\#intern--}{String.intern()} method as follows:

\begin{Verbatim}
  @Interned String intern();
\end{Verbatim}

  You may also remove irrelevant parts of the stub file; see
  Section~\ref{stub-format}.

\end{enumerate}


\subsection{Troubleshooting stub libraries\label{stub-troubleshooting}}


\subsubsection{Type-checking does not yield the expected results\label{stub-troubleshooting-type-checking-results}}

By default, the stub parser silently ignores
annotations on unknown classes and methods.
The stub parser also silently ignores unknown annotations, so don't forget to
\<import> any annotations.

Use command-line option
\<-AstubWarnIfNotFound> to warn whenever some element of a stub file cannot
be found.

The \<@NoStubParserWarning> annotation on a package or type in a stub file
overrides the \<-AstubWarnIfNotFound> command-line option, and no warning
will be issued.

Use command-line option
\<-AstubWarnIfOverwritesBytecode> to warn whenever some element of a
stub file overwrites annotations contained in bytecode.

Use command-line option \<-AstubDebug> to output debugging messages while
parsing stub files, including about unknown classes, methods, and
annotations.  This overrides the \<@NoStubParserWarning> annotation.



\subsubsection{Problems parsing stub libraries\label{stub-troubleshooting-parsing}}

When using command-line option \<-AstubWarnIfNotFound>,
an error is issued if a stub file has a typo or the API method does not
exist.

Fix this error by removing the extra L in the method name:
\begin{Verbatim}
StubParser: Method isLLowerCase(char) not found in type java.lang.Character
\end{Verbatim}

Fix this error by removing the method \<enableForgroundNdefPush(...)> from
the stub file, because it is not defined in class \<android.nfc.NfcAdapter>
in the version of the library you are using:
\begin{Verbatim}
StubParser: Method enableForegroundNdefPush(Activity,NdefPushCallback)
      not found in type android.nfc.NfcAdapter
\end{Verbatim}


\section{Library annotations should reflect the specification, not the implementation\label{libraries-specification}}

Publicly-visible annotations (including those on public method formal
parameters and return types) should be based on the
documentation, typically the method's Javadoc.

If a fact is not mentioned in the documentation, then it is an
implementation detail.  Clients should not depend on such implementation
details, which are prone to change without notice.

If there is a fact that you think should be in the library's documentation
but was unintentionally omitted by its authors, then please submit a bug
report asking them to update the documentation to reflect this fact.  If
they do, then you can also express the fact as an annotation.

If you wish to depend on your assumption that the behavior will never
change, then you can create a local stub file as a workaround while you
wait for the library documentation to be updated.


\section{Troubleshooting/debugging annotated libraries\label{libraries-troubleshooting}}

Sometimes, it may seem that a checker is treating a library as unannotated
even though the library has annotations.  The compiler has two flags that
may help you in determining whether library files are read, and if they are
read whether the library's annotations are parsed.

\begin{description}
\item \<-verbose>
  Outputs info about compile phases --- when the compiler
  reads/parses/attributes/writes any file.  Also outputs the classpath and
  sourcepath paths.
\item \<-XDTA:parser> (which is equivalent to \<-XDTA:reader> plus \<-XDTA:writer>)
  Sets the internal \<debugJSR308> flag, which outputs information about
  reading and writing.
\end{description}


% LocalWords:  plugin utils util dist RuntimeException NonNull TODO AFU enum
% LocalWords:  sourcepath Nullness javac classpath src quals pathSeparator JDKs
% LocalWords:  jdk Astubs skipUses astub AskipUses toArray JDK6 xvzf javax
% LocalWords:  CollectionToArrayHeuristics BaseTypeVisitor Xbootclasspath
% LocalWords:  Interning's UsesObjectEquals Anocheckjdk AonlyUses java pre
%  LocalWords:  Aignorejdkastub AstubWarnIfNotFound AstubDebug jdk6 jdk7
%  LocalWords:  enableForgroundNdefPush XDTA debugJSR308 BCEL getopt jdk8
%%  LocalWords:  NoStubParserWarning CHECKERFRAMEWORK AnnotatedFor regex
%%  LocalWords:  AuseConservativeDefaultsForUnannotatedCode buildfile qual
%%  LocalWords:  AprintUnannotatedMethods checkername AskipDefs bcel mkdir
%%  LocalWords:  AuseSafeDefaultsForUnannotatedSourceCode TypeSystem1 cd
%%  LocalWords:  TypeSystem2 TypeSystem3 AuseDefaultsForUncheckedCode ln
%  LocalWords:  mychecker DIRS README TypeSystem un debugJSR org boolean
%  LocalWords:  AstubWarnIfOverwritesBytecode Awarns
