\htmlhr
\chapterAndLabel{Purity Checker}{purity-checker}

The Purity Checker identifies methods that have no side effects, that
return the same value each time they are called on the same argument, or both.

Purity analysis aids type refinement (Section~\ref{type-refinement}).

All checkers utilize purity annotations on called methods.
You do not need to run the Purity Checker directly.
However you may run just the Purity Checker by
supplying the following command-line options to javac:
\code{-processor org.checkerframework.util.PurityChecker}.


\sectionAndLabel{Purity annotations}{purity-annotations}

\begin{description}

\item[\refqualclass{dataflow/qual}{SideEffectFree}]
  indicates that the method has no externally-visible side effects.

\item[\refqualclass{dataflow/qual}{Deterministic}]
  indicates that if the method is called multiple times with identical
  arguments, then it returns the identical result according to \<==>
  (not just according to \<equals()>).

\item[\refqualclass{dataflow/qual}{Pure}]
  indicates that the method is both \<@SideEffectFree> and \<@Deterministic>.

\end{description}

If you supply the command-line option \<-AsuggestPureMethods>, then the
Checker Framework will suggest methods that can be marked as
\<@SideEffectFree>, \<@Deterministic>, or \<@Pure>.


\sectionAndLabel{Purity annotations are trusted}{purity-trusted}

Currently, purity annotations are trusted.  Purity annotations on called
methods affect type-checking of client code.  However, you can make a
mistake by writing \<@SideEffectFree> on the declaration of a method that
is not actually side-effect-free or by writing \<@Deterministic> on the
declaration of a method that is not actually deterministic.

To enable
checking of the annotations, supply the command-line option
\<-AcheckPurityAnnotations>.  It is not enabled by default because of a high false
positive rate.  In the future, after a new purity-checking analysis is
implemented, the Checker Framework will default to checking purity
annotations.


\sectionAndLabel{Overriding methods must respect specifications in superclasses}{purity-override}

If a method in a superclass has a purity annotation, then every overriding
definition must also have that purity annotation (or a stronger one).

Here is an example error if this requirement is violated:

\begin{mysmall}
\begin{Verbatim}
MyClass.java:1465: error: int hashCode() in MyClass cannot override int hashCode(Object this) in java.lang.Object;
attempting to use an incompatible purity declaration
    public int hashCode() {
               ^
  found   : []
  required: [SIDE_EFFECT_FREE, DETERMINISTIC]
\end{Verbatim}
\end{mysmall}

\noindent
The reason for the error is that the \<Object> class is annotated as:

\begin{Verbatim}
class Object {
  ...
  @Pure int hashCode() { ... }
}
\end{Verbatim}

\noindent
(where \refqualclass{dataflow/qual}{Pure} means both
\refqualclass{dataflow/qual}{SideEffectFree} and
\refqualclass{dataflow/qual}{Deterministic}).  Every overriding
definition, including those in your program, must use be at least as strong
a specification.  For \<hashCode>, every overriding definition must be
annotated as \<@Pure>.

You can fix the definition by adding \<@Pure> to your method definition.
Alternately, you can suppress the warning.
You can suppress each such warning individually using
\<@SuppressWarnings("purity.invalid.overriding")>,
or you can use the \<-AsuppressWarnings=purity.invalid.overriding>
command-line argument to suppress all such warnings.
% The \<-AassumeSideEffectFree> command-line argument suppresses the
% warning, but it does more too; don't mention it here.
In the future, the Checker Framework will support inheriting annotations
from superclass definitions.


\sectionAndLabel{Suppressing warnings}{purity-suppress-warnings}

The command-line option
\<-AassumeSideEffectFree> makes the Checker Framework unsoundly
assume that \emph{every} called method is side-effect-free.  This can make
flow-sensitive type refinement much more effective, since method calls will
not cause the analysis to discard information that it has learned.
However, this option can mask real errors.  It is most appropriate when you
are starting out annotating a project, or if you are using the Checker
Framework to find some bugs but not to give a guarantee that no more errors
exist of the given type.

%%  LocalWords:  AsuggestPureMethods AcheckPurityAnnotations
%%  LocalWords:  AsuppressWarnings AassumeSideEffectFree
