\htmlhr
\chapterAndLabel{Integration with external tools}{external-tools}

This chapter discusses how to run a checker from the command line, from a
build system, or from an IDE\@.  You can skip to the appropriate section:

% Keep this list up to date with the sections of this chapter and with a
% copy of the list in file introduction.tex .
\begin{itemize}
\item Android (Section~\ref{android})
\item Android Gradle Plugin (Section~\ref{android-gradle})
\item Ant (Section~\ref{ant-task})
\item Buck (Section~\ref{buck})
\item Command line, via Checker Framework javac wrapper (Section~\ref{javac-wrapper})
\item Command line, via JDK javac (Section~\ref{javac})
\item Eclipse (Section~\ref{eclipse})
\item Gradle (Section~\ref{gradle})
\item IntelliJ IDEA (Section~\ref{intellij})
\item javac Diagnostics Wrapper (Section~\ref{javac-diagnostics-wrapper})
\item Lombok (Section~\ref{lombok})
\item Maven (Section~\ref{maven})
\item NetBeans (Section~\ref{netbeans})
\item sbt (Section~\ref{sbt})
\item tIDE (Section~\ref{tide})
\end{itemize}

If your build system or IDE is not listed above, you should customize how
it runs the javac command on your behalf.  See your build system or IDE
documentation to learn how to
customize it, adapting the instructions for javac in Section~\ref{javac}.
If you make another tool support running a checker, please
inform us via the
\href{https://groups.google.com/forum/#!forum/checker-framework-discuss}{mailing
  list} or
\href{https://github.com/typetools/checker-framework/issues}{issue tracker} so
we can add it to this manual.

All examples in this chapter are in the public domain, with no copyright nor
licensing restrictions.


\sectionAndLabel{Android}{android}

When creating an Android app, you may wish to use \<checker-qual-android>
whenever this document mentions \<checker-qual>.  This can lead to smaller
dex files (smaller distributed apps).

The \<checker-qual-android> artifact is identical to the \<checker-qual>
artifact, except that in \<checker-qual-android> annotations have classfile
retention.  The default Android Gradle plugin retains types annotated with
runtime-retention annotations in the main dex, but strips out class-retention
annotations.


\sectionAndLabel{Android Studio and the Android Gradle Plugin}{android-gradle}

Android Studio 3.0 and later, and Android Gradle Plugin 3.0.0 and later, support type
annotations.  (See
\url{https://developer.android.com/studio/write/java8-support}
for more details.)  This section explains how to configure your Android
project to use the Checker Framework.  All the changes should be made to
the module's \<build.gradle> file --- not the project's \<build.gradle> file.

Different changes are required for JDK 8
(Section~\ref{android-jdk8}) and for JDK 9+ (Section~\ref{android-jdk11}).

\subsectionAndLabel{JDK 8}{android-jdk8}
This section shows what changes to make if you are using JDK 8.

\begin{enumerate}

\item In your module's \<build.gradle> file, set the source and target
  compatibility to \<JavaVersion.VERSION\_1\_8>:

\begin{Verbatim}
android {
    ...
    compileOptions {
        sourceCompatibility JavaVersion.VERSION_1_8
        targetCompatibility JavaVersion.VERSION_1_8
    }
}
\end{Verbatim}

\item Add a build variant for running checkers:

 \begin{Verbatim}
 android {
    ...
      buildTypes {
      ...
        checkTypes {
            javaCompileOptions.annotationProcessorOptions.
                    classNames.add("org.checkerframework.checker.nullness.NullnessChecker")
            // You can pass options like so:
            // javaCompileOptions.annotationProcessorOptions.arguments.put("warns", "")
        }
    }
}
\end{Verbatim}

\item Add a dependency configuration for the Java 9 compiler:

\begin{mysmall}
\begin{Verbatim}
configurations {
    errorproneJavac {
        description = 'Java 9 compiler; required to run the Checker Framework under JDK 8.'
    }
}

\end{Verbatim}
\end{mysmall}

\item Declare the Checker Framework dependencies:

\begin{mysmall}
\begin{Verbatim}
dependencies {
    ... existing dependencies...
    ext.checkerFrameworkVersion = '3.31.0'
    implementation "org.checkerframework:checker-qual-android:${checkerFrameworkVersion}"
    // or if you use no annotations in source code the above line could be
    // compileOnly "org.checkerframework:checker-qual-android:${checkerFrameworkVersion}"
    annotationProcessor "org.checkerframework:checker:${checkerFrameworkVersion}"
    errorproneJavac 'com.google.errorprone:javac:9+181-r4173-1'
}
\end{Verbatim}
\end{mysmall}

\item Direct all tasks of type \<JavaCompile> used by the \<checkTypes>
  build variant to use the Error Prone compiler:
\begin{mysmall}
\begin{Verbatim}
gradle.projectsEvaluated {
    tasks.withType(JavaCompile).all { compile ->
        if (compile.name.contains("CheckTypes")) {
            options.fork = true
            options.forkOptions.jvmArgs += ["-Xbootclasspath/p:${configurations.errorproneJavac.asPath}".toString()]
        }
    }
}
\end{Verbatim}
\end{mysmall}

\item To run the checkers, build using the \<checkTypes> variant:
\begin{Verbatim}
gradlew assembleCheckTypes
\end{Verbatim}
\end{enumerate}

\subsectionAndLabel{JDK 11+}{android-jdk11}
This section shows what changes to make if you are using JDK 11 or later.
% We have tested them with JDK 11.

\begin{enumerate}

\item In your module's \<build.gradle> file, set the source and target
  compatibility to \<JavaVersion.VERSION\_1\_8>:

\begin{Verbatim}
android {
    ...
    compileOptions {
        sourceCompatibility JavaVersion.VERSION_1_8
        targetCompatibility JavaVersion.VERSION_1_8
    }
}
\end{Verbatim}

\item Add a build variant for running checkers:

 \begin{Verbatim}
 android {
    ...
      buildTypes {
      ...
        checkTypes {
            javaCompileOptions.annotationProcessorOptions.
                    classNames.add("org.checkerframework.checker.nullness.NullnessChecker")
            // You can pass options like so:
            // javaCompileOptions.annotationProcessorOptions.arguments.put("warns", "")
        }
    }
}
\end{Verbatim}

\item Declare the Checker Framework dependencies:

\begin{mysmall}
\begin{Verbatim}
dependencies {
    ... existing dependencies...
    ext.checkerFrameworkVersion = '3.31.0'
    implementation "org.checkerframework:checker-qual-android:${checkerFrameworkVersion}"
    // or if you use no annotations in source code the above line could be
    // compileOnly "org.checkerframework:checker-qual-android:${checkerFrameworkVersion}"
    annotationProcessor "org.checkerframework:checker:${checkerFrameworkVersion}"
}
\end{Verbatim}
\end{mysmall}

\item To run the checkers, build using the \<checkTypes> variant:
\begin{Verbatim}
gradlew assembleCheckTypes
\end{Verbatim}

\end{enumerate}


\sectionAndLabel{Ant task}{ant-task}

If you use the \href{https://ant.apache.org/}{Ant} build tool to compile
your software, then you can add an Ant task that runs a checker.  We assume
that your Ant file already contains a compilation target that uses the
\code{javac} task, and that the \<CHECKERFRAMEWORK> environment variable is set.

\begin{enumerate}
\item
Set the \code{cfJavac} property:

%BEGIN LATEX
\begin{smaller}
%END LATEX
\begin{Verbatim}
  <property environment="env"/>
  <property name="checkerframework" value="${env.CHECKERFRAMEWORK}" />
  <condition property="cfJavac" value="javac.bat" else="javac">
    <os family="windows" />
  </condition>
  <presetdef name="cf.javac">
    <javac fork="yes" executable="${checkerframework}/checker/bin/${cfJavac}" >
      <compilerarg value="-version"/>
      <compilerarg value="-implicit:class"/>
    </javac>
  </presetdef>
\end{Verbatim}
%BEGIN LATEX
\end{smaller}
%END LATEX

\item \textbf{Duplicate} the compilation target, then \textbf{modify} it slightly as
indicated in this example:

%BEGIN LATEX
\begin{smaller}
%END LATEX
\begin{Verbatim}
  <target name="check-nullness"
          description="Check for null pointer dereferences"
          depends="clean,...">
    <!-- use cf.javac instead of javac -->
    <cf.javac ... >
      <compilerarg line="-processor org.checkerframework.checker.nullness.NullnessChecker"/>
      <!-- optional, to not check uses of library methods:
        <compilerarg value="-AskipUses=^(java\.awt\.|javax\.swing\.)"/>
      -->
      <compilerarg line="-Xmaxerrs 10000"/>
      ...
    </cf.javac>
  </target>
\end{Verbatim}
%BEGIN LATEX
\end{smaller}
%END LATEX

Fill in each ellipsis (\ldots) from the original compilation target.
However, do not copy any \<fork=> setting from the original \code{<javac>}
task invocation.

If your original compilation target set the bootclasspath, then you cannot
use the javac wrapper script as the above instructions do.  You should edit
your Ant buildfile to make invocations similar to those described in
Section~\ref{javac}, but accommodating your bootclasspath.

In the example, the target is named \code{check-nullness}, but you can
name it whatever you like.
\end{enumerate}

\subsectionAndLabel{Explanation}{ant-task-explanation}

This section explains each part of the Ant task.

\begin{enumerate}
\item Definition of \code{cf.javac}:

The \code{fork} field of the \code{javac} task
ensures that an external \code{javac} program is called.  Otherwise, Ant will run
\code{javac} via a Java method call, and there is no guarantee that it will get
correct version of \code{javac}.

The \code{-version} compiler argument is just for debugging; you may omit
it.

The \code{-implicit:class} compiler argument causes annotation processing
to be performed on implicitly compiled files.  (An implicitly compiled file
is one that was not specified on the command line, but for which the source
code is newer than the \code{.class} file.)  This is the default, but
supplying the argument explicitly suppresses a compiler warning.

%% -Awarns was removed above without removing it here.
% The \code{-Awarns} compiler argument is optional, and causes the checker to
% treat errors as warnings so that compilation does not fail even if
% pluggable type-checking fails; see Section~\ref{checker-options}.

\item The \code{check-nullness} target:

The target assumes the existence of a \code{clean} target that removes all
\code{.class} files.  That is necessary because Ant's \code{javac} target
doesn't re-compile \code{.java} files for which a \code{.class} file
already exists.

The \code{-processor ...} compiler argument indicates which checker to
run.  You can supply additional arguments to the checker as well.

\end{enumerate}


\sectionAndLabel{Buck}{buck}

Buck is a build system maintained by Facebook.

Buck has support for annotation processors, but that support is
undocumented because the Buck maintainers may change the syntax in the
future and they don't wish to ever change anything that is documented.
You can learn more about Buck and annotation processors at these URLs:
{\codesize\url{https://stackoverflow.com/questions/32915721/documentation-for-annotation-processors-buck}},
{\codesize\url{https://github.com/facebook/buck/issues/85}}.

In order to use Checker Framework with Buck on \textbf{JDK 8}, you first need
to place the Error Prone JDK 9 compiler in Buck's bootclasspath
(further explanation in Section~\ref{javac-jdk8}).  To do so,
follow the instructions on this page: \\
{\codesize\url{https://github.com/uber/okbuck/wiki/Using-Error-Prone-with-Buck-and-OkBuck#using-error-prone-javac-on-jdk-8}} \\
You only need to follow the instructions to use Error Prone javac on
that page, not the instructions to enable Error Prone.

Here is an example \<BUCK> build
file showing how to enable the Checker Framework:

\begin{Verbatim}
prebuilt_jar(
    name = 'checker-framework',
    binary_jar = 'checker-3.31.0.jar',
    visibility = [ 'PUBLIC' ]
)

prebuilt_jar(
    name = 'checker-qual',
    binary_jar = 'checker-qual-3.31.0.jar',
    visibility = [ 'PUBLIC' ]
)

java_library (
    name = 'hello',
    srcs = glob(['src/main/java/**/*.java']),
    java_version = '1.8',
    provided_deps = [ ':checker-framework', ':checker-qual' ],
# To add annotation processing
    annotation_processors = [ 'org.checkerframework.checker.units.UnitsChecker' ],
    annotation_processor_deps = [ ':checker-framework', ':checker-qual' ],
)
\end{Verbatim}


\subsectionAndLabel{Troubleshooting}{buck-troubleshooting}

Use \<--verbose 8> to see diagnostic output including a command line that
ought to be equivalent to Buck's use of an in-process Java compiler.
For example, you might run

\begin{Verbatim}
./buck clean
./buck fetch //...
./buck build --verbose 8 //app:main
\end{Verbatim}

However, the command lines that buck prints with \<-v 8> is \emph{not} what
Buck actually executes.  Buck's invocation of the Checker Framework may
produce errors that cannot be reproduced from the command line.  This is
because Buck uses
\href{https://github.com/facebook/buck/blob/main/src/com/facebook/buck/jvm/java/Jsr199Javac.java}{in-process
  compilation}, potentially multi-threaded, and because Buck may use
\href{https://github.com/facebook/buck/blob/main/src/com/facebook/buck/util/ClassLoaderCache.java}{multiple
  classloaders} that are cached.

You can force Buck to run an external javac, thus behaving exactly like the
command line that its diagnostics print.  Use the configuration flag \<-c
tools.javac=\emph{path-to-javac}>, for example:

\begin{Verbatim}
./buck clean && ./buck fetch //... && ./buck build //app:main -c tools.javac=$(which javac)
\end{Verbatim}


\sectionAndLabel{Command line, via Checker Framework javac wrapper}{javac-wrapper}
\label{javac-installation}      % for backward compatibility, added 10/2/2019

To perform pluggable type-checking from the command line, run the \<javac>
command that ships with the Checker Framework.  This is called the
``Checker Framework compiler'' or the ``Checker Framework \<javac> wrapper''.
It is exactly the same as the OpenJDK
\<javac> compiler, with one small difference:  it includes the Checker
Framework jar file on its classpath.

You cannot use the \<javac> wrapper if you wish to set the bootclasspath.

There are three ways to use the Checker Framework compiler from the command
line.  You can use any
one of them.  However, if you are using the Windows command shell, you must
use the last one.
% Is the last one required for Cygwin, as well as for the Windows command shell?
Adjust the pathnames if you have installed the Checker Framework somewhere
other than \<\${HOME}/checker-framework-3.31.0/>.


\begin{itemize}
  \item
    Option 1:
    Add directory
    \code{.../checker-framework-3.31.0/checker/bin} to your path, \emph{before} any other
    directory that contains a \<javac> executable.

    If you are
    using the bash shell, a way to do this is to add the following to your
    \verb|~/.profile| (or alternately \verb|~/.bash_profile| or \verb|~/.bashrc|) file:
\begin{Verbatim}
  export CHECKERFRAMEWORK=${HOME}/checker-framework-3.31.0
  export PATH=${CHECKERFRAMEWORK}/checker/bin:${PATH}
\end{Verbatim}

   After editing the file, log out and back in to ensure that the environment variable
   setting takes effect.

  \item
    \begin{sloppypar}
    Option 2:
    Whenever this document tells you to run \code{javac},
    instead run \code{\$CHECKERFRAMEWORK/checker/bin/javac}.
    \end{sloppypar}

    You can simplify this by introducing an alias \<javacheck>.  Then,
    whenever this document tells you to run \code{javac}, instead run
    \<javacheck>.  Here is the syntax for your
    \verb|~/.bashrc|, \verb|~/.profile|, or \verb|~/.bash_profile|
    file:
% No Windows example because this doesn't work under Windows.
\begin{Verbatim}
  export CHECKERFRAMEWORK=${HOME}/checker-framework-3.31.0
  alias javacheck='$CHECKERFRAMEWORK/checker/bin/javac'
\end{Verbatim}

   After editing the file, log out and back in to ensure that the environment variable
   setting and alias take effect.

   \item
   Option 3:
   Whenever this document tells you to run \code{javac}, instead
   run \<checker.jar> via \<java> (not \<javac>) as in:

\begin{Verbatim}
  java -jar "$CHECKERFRAMEWORK/checker/dist/checker.jar" -cp "myclasspath" -processor nullness MyFile.java
\end{Verbatim}

    You can simplify the above command by introducing an alias
    \<javacheck>.  Then, whenever this document tells you to run
    \code{javac}, instead run \<javacheck>.  For example:

\begin{Verbatim}
  # Unix
  export CHECKERFRAMEWORK=${HOME}/checker-framework-3.31.0
  alias javacheck='java -jar "$CHECKERFRAMEWORK/checker/dist/checker.jar"'

  # Windows
  set CHECKERFRAMEWORK = C:\Program Files\checker-framework-3.31.0\
  doskey javacheck=java -jar "%CHECKERFRAMEWORK%\checker\dist\checker.jar" $*
\end{Verbatim}

  (Explanation for advanced users:
  More generally, anywhere that you would use \<javac.jar>, you can substitute
  \<\$CHECKERFRAMEWORK/checker/dist/checker.jar>;
  the result is to use the Checker
  Framework compiler instead of the regular \<javac>.)

\end{itemize}



%% Does this work?  Text elsewhere in the manual imples that it does not.
% \item
% \begin{sloppypar}
%   In order to use the updated compiler when you type \code{javac}, add the
%   directory \<C:\ttbs{}Program Files\ttbs{}checker-framework\ttbs{}checkers\ttbs{}binary> to the
%   beginning of your path variable.  Also set a \code{CHECKERFRAMEWORK} variable.
% \end{sloppypar}
%
% % Instructions stolen from http://www.webreference.com/js/tips/020429.html
%
% To set an environment variable, you have two options:  make the change
% temporarily or permanently.
% \begin{itemize}
% \item
% To make the change \textbf{temporarily}, type at the command shell prompt:
%
% \begin{alltt}
% path = \emph{newdir};%PATH%
% \end{alltt}
%
% For example:
%
% \begin{Verbatim}
% set CHECKERFRAMEWORK = C:\Program Files\checker-framework
% path = %CHECKERFRAMEWORK%\checker\bin;%PATH%
% \end{Verbatim}
%
% This is a temporary change that endures until the window is closed, and you
% must re-do it every time you start a new command shell.
%
% \item
% To make the change \textbf{permanently},
% Right-click the \<My Computer> icon and
% select \<Properties>. Select the \<Advanced> tab and click the
% \<Environment Variables> button. You can set the variable as a ``System
% Variable'' (visible to all users) or as a ``User Variable'' (visible to
% just this user).  Both work; the instructions below show how to set as a
% ``System Variable''.
% In the \<System Variables> pane, select
% \<Path> from the list and click \<Edit>. In the \<Edit System Variable>
% dialog box, move the cursor to the beginning of the string in the
% \<Variable Value> field and type the full directory name (not using the
% \verb|%CHECKERFRAMEWORK%| environment variable) followed by a
% semicolon (\<;>).
%
% % This is for the benefit of the Ant task.
% Similarly, set the \code{CHECKERFRAMEWORK} variable.
%
% This is a permanent change that only needs to be done once ever.
% \end{itemize}


\sectionAndLabel{Command line, via JDK javac}{javac}

This section explains how to use the Checker Framework with the OpenJDK or
OracleJDK \<javac>, rather than with the \<javac> wrapper script described in
Section~\ref{javac-wrapper}.

This
section assumes you have downloaded the Checker Framework release zip and set
the environment variable \<CHECKERFRAMEWORK> to point to the unzipped directory.
Alternately, you can get all of the jars mentioned in this section from Maven Central:

\begin{itemize}
\item \<javac.jar>: \url{https://search.maven.org/artifact/com.google.errorprone/javac/9%2B181-r4173-1/jar}
\item \<checker-qual.jar>: \url{https://repo1.maven.org/maven2/org/checkerframework/checker-qual/3.31.0/checker-qual-3.31.0.jar}
\item \<checker-util.jar>: \url{https://repo1.maven.org/maven2/org/checkerframework/checker-util/3.31.0/checker-util-3.31.0.jar}
\item \<checker.jar>: \url{https://repo1.maven.org/maven2/org/checkerframework/checker/3.31.0/checker-3.31.0-all.jar}
\end{itemize}

Different arguments to \<javac> are required for JDK 8
(Section~\ref{javac-jdk8}) and for JDK 9+ (Section~\ref{javac-jdk11}).


\subsectionAndLabel{JDK 8}{javac-jdk8}

This section shows what arguments to pass to the OpenJDK or OracleJDK
\<javac> (\emph{not} the \<javac> wrapper script of
Section~\ref{javac-wrapper}) to run the Checker
Framework, if you are using JDK 8.

\begin{Verbatim}
javac \
-J-Xbootclasspath/p:$CHECKERFRAMEWORK/checker/dist/javac.jar \
-cp $CHECKERFRAMEWORK/checker/dist/checker-qual.jar \
-processorpath $CHECKERFRAMEWORK/checker/dist/checker.jar \
-processor org.checkerframework.checker.nullness.NullnessChecker \
-source 8 -target 8
\end{Verbatim}

Below is an explanation of each argument.
\begin{enumerate}
\item \<-J-Xbootclasspath/p:\$CHECKERFRAMEWORK/checker/dist/javac.jar>:
even when running under JDK 8, a Java 9+ compiler (either the Error Prone
compiler or the OpenJDK Java compiler (version 9 or later) must be on the JVM bootclasspath.
The \<-J> indicates that this is a JVM argument rather than a compiler
argument.

The following exception is thrown if this argument is missing:
\begin{Verbatim}
error: SourceChecker.typeProcessingStart: unexpected Throwable (NoSuchMethodError);
message: com.sun.tools.javac.code.Type.stripMetadata()Lcom/sun/tools/javac/code/Type;
\end{Verbatim}

If you are compiling your own checker, the following exception is thrown if this argument is missing:
\begin{Verbatim}
java.lang.NoSuchFieldError: ANNOTATION_PROCESSOR_MODULE_PATH
\end{Verbatim}

\item \<-cp \$CHECKERFRAMEWORK/checker/dist/checker-qual.jar>: \<checker-qual.jar>
must be on the compilation classpath.

\item \<-processorpath \$CHECKERFRAMEWORK/checker/dist/checker.jar>:
\<checker.jar> must be on the processor path. Using this option means that the processor path, not
the classpath, will be searched for annotation processors
and for classes that they load.

\item \<-processor org.checkerframework.checker.nullness.NullnessChecker>:
Choose which checker to run by passing its fully qualified name as a processor.
(Note, using this option means that javac will not search for annotation
processors, but rather will run only those specified here.)

\item \<-source 8 -target 8>: Because the build is using
a Java 9 compiler, these options ensure that the
source code is Java 8 and that Java 8 bytecode is created.

\end{enumerate}


\subsectionAndLabel{JDK 11+}{javac-jdk11}

This section shows what arguments to pass to the OpenJDK or OracleJDK
\<javac> (\emph{not} the \<javac> wrapper script of
Section~\ref{javac-wrapper}) to run the Checker
Framework, if you are using JDK 9 or later.  These
instructions should work on JDK 9 or later, but we only test with supported
versions of Java: JDK 11, JDK 17, and JDK 19.


\subsubsectionAndLabel{Non-modularized code}{javac-jdk11-non-modularized}

To compile non-modularized code:

% Keep list of --add-opens and --add-exports in sync with the "maven" and "SBT" sections,
% the method CheckerMain#getExecArguments, checker-framework/build.gradle in compilerArgsForRunningCFs,
% and in the checker-framework-gradle-plugin, CheckerFrameworkPlugin#applyToProject
\begin{Verbatim}
javac \
-J--add-exports=jdk.compiler/com.sun.tools.javac.api=ALL-UNNAMED \
-J--add-exports=jdk.compiler/com.sun.tools.javac.code=ALL-UNNAMED \
-J--add-exports=jdk.compiler/com.sun.tools.javac.file=ALL-UNNAMED \
-J--add-exports=jdk.compiler/com.sun.tools.javac.main=ALL-UNNAMED \
-J--add-exports=jdk.compiler/com.sun.tools.javac.model=ALL-UNNAMED \
-J--add-exports=jdk.compiler/com.sun.tools.javac.processing=ALL-UNNAMED \
-J--add-exports=jdk.compiler/com.sun.tools.javac.tree=ALL-UNNAMED \
-J--add-exports=jdk.compiler/com.sun.tools.javac.util=ALL-UNNAMED \
-J--add-opens=jdk.compiler/com.sun.tools.javac.comp=ALL-UNNAMED \
-processorpath $CHECKERFRAMEWORK/checker/dist/checker.jar \
-cp $CHECKERFRAMEWORK/checker/dist/checker-qual.jar \
-processor org.checkerframework.checker.nullness.NullnessChecker
\end{Verbatim}

The arguments are explained above, except for two of them:
\begin{enumerate}

\item \<-J--add-opens=jdk.compiler/com.sun.tools.javac.comp=ALL-UNNAMED> which
opens the \<jdk.compiler/com.sun.tools.java.comp> package.  This is
required because the Checker Framework reflectively accesses private members of this package.

\item \<-J--add-exports=...=ALL-UNNAMED> which exports the listed packages.  This is
required with JDK 17+ because the Checker Framework accesses non-exported members of these packages.
(These are also required with JDK 11 if \<--illegal-access> is set to \<deny>.)

\end{enumerate}

If this option is missing, \<javac> may issue one of the below warnings:
\begin{Verbatim}
WARNING: An illegal reflective access operation has occurred
WARNING: Illegal reflective access by org.checkerframework.javacutil.Resolver
  (file:$CHECKERFRAMEWORK/checker/dist/checker.jar) to method com.sun.tools.javac.comp.Resolve.findMethod(...)
WARNING: Please consider reporting this to the maintainers of org.checkerframework.javacutil.Resolver
WARNING: Use --illegal-access=warn to enable warnings of further illegal reflective access operations
WARNING: All illegal access operations will be denied in a future release
\end{Verbatim}

\begin{Verbatim}
java.lang.IllegalAccessError: class org.checkerframework.javacutil.AbstractTypeProcessor
  (in unnamed module ...) cannot access class com.sun.tools.javac.processing.JavacProcessingEnvironment
  (in module jdk.compiler) because module jdk.compiler does not export com.sun.tools.javac.processing
  to unnamed module ...
\end{Verbatim}


\subsubsectionAndLabel{Modularized code}{javac-jdk11-modularized}

To compile a module, first add \<requires
org.checkerframework.checker.qual;> to your \<module-info.java>.  The Checker
Framework inserts inferred annotations into bytecode even if none appear in source code,
so you must do this even if you write no annotations in your code.

\begin{Verbatim}
javac \
  -J--add-exports=jdk.compiler/com.sun.tools.javac.api=ALL-UNNAMED \
  -J--add-exports=jdk.compiler/com.sun.tools.javac.code=ALL-UNNAMED \
  -J--add-exports=jdk.compiler/com.sun.tools.javac.file=ALL-UNNAMED \
  -J--add-exports=jdk.compiler/com.sun.tools.javac.main=ALL-UNNAMED \
  -J--add-exports=jdk.compiler/com.sun.tools.javac.model=ALL-UNNAMED \
  -J--add-exports=jdk.compiler/com.sun.tools.javac.processing=ALL-UNNAMED \
  -J--add-exports=jdk.compiler/com.sun.tools.javac.tree=ALL-UNNAMED \
  -J--add-exports=jdk.compiler/com.sun.tools.javac.util=ALL-UNNAMED \
  -J--add-opens=jdk.compiler/com.sun.tools.javac.comp=ALL-UNNAMED \
  -processorpath $CHECKERFRAMEWORK/checker/dist/checker.jar \
  --module-path $CHECKERFRAMEWORK/checker/dist/checker-qual.jar \
  -processor org.checkerframework.checker.nullness.NullnessChecker
\end{Verbatim}

\<checker-qual.jar> must be on the module path rather than the class path, but
do not put \<checker.jar> on the processor module path as it is not
modularized.


\sectionAndLabel{Eclipse}{eclipse}

% Eclipse supports type annotations.
% Eclipse does not directly support running the Checker Framework,
% nor is Eclipse necessary for running the Checker Framework.

You
need to run the Checker Framework via a build tool (Ant, Gradle, Maven, etc.), rather
than by supplying the \<-processor> command-line option to the \<ejc>
compiler, which is also known as \<eclipsec>.
The reason is that the Checker Framework is built upon \<javac>,
and \<ejc> represents the Java program differently.  (If both \<javac> and \<ejc>
implemented JSR 198~\cite{JSR198}, then it would be possible to build
an annotation processor that works with both compilers.)


There is no dedicated Eclipse plug-in for running the Checker Framework,
but it's still easy to run the Checker Framework.  First, create a
target/task in your build system to run the Checker Framework.  Then, run
the target/task from Eclipse.  Section~\ref{eclipse-ant} gives details for
Ant, but other build systems are similar.


\subsectionAndLabel{Using an Ant task}{eclipse-ant}

Add an Ant target as described in Section~\ref{ant-task}.  You can
run the Ant target by executing the following steps
(instructions copied from
{\codesize\url{https://help.eclipse.org/luna/index.jsp?topic=%2Forg.eclipse.platform.doc.user%2FgettingStarted%2Fqs-84_run_ant.htm}}):

\begin{enumerate}

\item
  Select \code{build.xml} in one of the navigation views and choose
  {\bf Run As $>$ Ant Build...} from its context menu.

\item
  A launch configuration dialog is opened on a launch configuration
  for this Ant buildfile.

\item
  In the {\bf Targets} tab, select the new ant task (e.g., check-interning).

\item
  Click {\bf Run}.

\item
  The Ant buildfile is run, and the output is sent to the Console view.

\end{enumerate}


\subsectionAndLabel{Troubleshooting Eclipse}{eclipse-troubleshooting}

Eclipse issues an ``Unhandled Token in @SuppressWarnings'' warning if you
write a \<@SuppressWarnings> annotation containing a string that Eclipse does not
know about.  Unfortunately, Eclipse
\href{https://bugs.eclipse.org/bugs/show_bug.cgi?id=122475}{hard-codes}
this list.

To eliminate the warnings:
disable all ``Unhandled Token in @SuppressWarnings'' warnings in Eclipse.
Look under the menu headings
``Java $\rightarrow$ Compiler $\rightarrow$ Errors/Warnings $\rightarrow$
Annotations $\rightarrow$ Unhandled Token in '@SuppressWarnings',''
and set it to ignore.


\sectionAndLabel{Gradle}{gradle}

To run a checker
on a project that uses the \href{https://gradle.org/}{Gradle} build system,
use the
\ahreforurl{https://github.com/kelloggm/checkerframework-gradle-plugin}{Checker
  Framework Gradle plugin}.  Its documentation explains how to use it.


\sectionAndLabel{IntelliJ IDEA}{intellij}

% The following method has been tested with IntelliJ IDEA 2019.1.1.

This section tells you how to make IntelliJ IDEA automatically run a
checker on every compile for Java projects and/or modules.

If your project uses a build tool (Ant, Gradle, Maven, etc.), \emph{do not}
use the instructions in this section.
Follow the instructions for that build tool instead (they are in
a different section of this chapter).  To compile your project, run the
build tool (possibly from within IntelliJ IDEA, possibly not).


\subsectionAndLabel{Running a checker on every IntelliJ compilation}{intellij-every-compilation}

If your project does not use a build tool, use the following instructions
to run a checker within IntelliJ IDEA on every compile:

\begin{enumerate}

\item Change the project SDK to 11 or later, as explained at
  \url{https://www.jetbrains.com/help/idea/sdk.html#change-project-sdk}.

\item Make sure the language level for your project is 8 or higher, as explained at
\url{https://www.jetbrains.com/help/idea/project-page.html}.

\item Create and configure an annotation profile, following the instructions at \url{https://www.jetbrains.com/help/idea/annotation-processors-support.html}.
When configuring the profile:
\begin{enumerate}
\item Add \<.../checker-framework/checker/dist/checker.jar> to the processor path.
\item Add checkers to be run during compilation by writing the
  fully-qualified name of the checker in the ``Processor FQ Name''
  section.  An example of a fully-qualified checker name is
  ``org.checkerframework.checker.nullness.NullnessChecker''.
\end{enumerate}

\item Add \<.../checker-framework/checker/dist/checker-qual.jar>, as a dependency to all
modules you wish to type check. (They should all have been associated with
the annotation profile above.)
Instructions appear at
\url{https://www.jetbrains.com/help/idea/creating-and-managing-projects.html}.

\end{enumerate}

Now, when you compile your code, the checker will be run.

It is necessary to manually inform the IDE via a plugin if an annotation
system adds any dependencies beyond those that normally exist in Java.
For information about the extension points, see
\url{https://youtrack.jetbrains.com/issue/IDEA-159286}.


\subsectionAndLabel{Running a checker on every IntelliJ change or save}{intellij-every-save}

To make IntelliJ compile on every change or save,
follow the instructions at
\url{https://www.jetbrains.com/help/idea/compiling-applications.html#auto-build}.

You can also configure IntelliJ to automatically save (and thus
automatically compile) your work periodically. Instructions appear at
\url{https://www.jetbrains.com/help/idea/system-settings.html#sync}.

\sectionAndLabel{javac diagnostics wrapper}{javac-diagnostics-wrapper}

The \href{https://github.com/eisopux/javac-diagnostics-wrapper}{javac
diagnostics wrapper} project can post-process the javac
diagnostics output into other formats, such as
the LSP (Language Server Protocol) JSON style.


\sectionAndLabel{Lombok}{lombok}

Project Lombok (\url{https://projectlombok.org/}) is a library that
generates getter, setter, and builder methods, among other features.
For example, if you declare
a field:

\begin{Verbatim}
  @Getter @Setter
  private @Regex String role;
\end{Verbatim}

\noindent
then Lombok will generate getter and setter methods:

\begin{Verbatim}
  public @Regex String getRole() { return role; }
  public void setRole(@Regex String role) { this.role = role; }
\end{Verbatim}


\subsectionAndLabel{Annotations on generated code}{lombok-copying-annotations}

As illustrated in the example above, Lombok copies type annotations from fields
to generated methods, when the user writes Lombok's \<@Getter>, \<@Setter>,
and \<@Builder> annotations.  Lombok does so only for certain type
annotations (including all annotations in the Checker Framework
distribution); see variable \<BASE\_COPYABLE\_ANNOTATIONS> in file
\href{https://github.com/projectlombok/lombok/blob/master/src/core/lombok/core/handlers/HandlerUtil.java}{\<HandlerUtil.java>}.

To make Lombok copy other type annotations from fields to generated code,
add those type annotations to the \<lombok.copyableAnnotations>
configuration key in your \<lombok.config> file.  For example:

\begin{Verbatim}
  lombok.copyableAnnotations += my.checker.qual.MyTypeAnnotation
\end{Verbatim}

Directory \<docs/examples/lombok> contains an example Gradle project that
augments the configuration key.

% Alternatives:
%  * It would be better if Lombok automatically copied all type annotations.
%    Unfortunately, there is no way for Lombok to know whether an annotation
%    is a type annotation, at the (early) point in the javac pipeline where
%    Lombok runs.

% TODO: write a simple tool that generates a lombok.config file for a particular
% project, with all type annotations used in the project.


\subsectionAndLabel{Type-checking code with Lombok annotations}{lombok-typechecking}

If you run the Checker Framework and Lombok in the same \<javac>
invocation, the Checker Framework cannot type-check a class that contains
Lombok annotations.  The way that Lombok changes the class prevents the
Checker Framework from seeing any of the class.  (The Checker Framework
works fine on classes that do not contain Lombok annotations, including if
they call Lombok-generated code.)

Therefore, you must run the Checker Framework in a postpass after the
\<javac> that runs Lombok has completed.  Use the
\href{https://projectlombok.org/features/delombok}{Delombok} tool
(distributed with Lombok) to generate Java source code, then run the
Checker Framework on that.  The
\href{https://github.com/kelloggm/checkerframework-gradle-plugin}{Checker
  Framework Gradle plugin} does this for you automatically.


\subsectionAndLabel{Maven Projects using Lombok}{lombok-maven}

\href{http://anthonywhitford.com/lombok.maven/lombok-maven-plugin/}{Lombok Maven Plugin}
can be used to delombok source code for Maven projects. The
\href{http://anthonywhitford.com/lombok.maven/lombok-maven-plugin/faq.html#alt-src-setup}
{Lombok Maven Plugin FAQ} discusses how to configure the plugin when code
using Lombok is mixed with standard Java code. Once the Lombok Maven plugin
is configured correctly, the Checker Framework Maven configuration can be added
independently.


\sectionAndLabel{Maven}{maven}

If you use the \href{https://maven.apache.org/}{Maven} tool,
then you can enable Checker Framework checkers by following the
instructions below.

See the directory \code{docs/examples/MavenExample/} for examples of the use of
Maven build files.
This example can be used to verify that
Maven is correctly downloading the Checker Framework from the
\href{https://search.maven.org/search?q=org.checkerframework}{Maven
  Central Repository} and executing it.

Please note that the \<-AoutputArgsToFile> command-line option
(see Section~\ref{creating-debugging-options-output-args}) and shorthands for built-in checkers
(see Section~\ref{shorthand-for-checkers}) are not available when
following these instructions.  Both these features are available only when a checker is
launched via \<checker.jar> such as when \code{\$CHECKERFRAMEWORK/checker/bin/javac}
is run.  The instructions in this section
bypass \<checker.jar> and cause the compiler to run a
checker as an annotation processor directly.

Debugging your Maven configuration can be tricky because of a
\ahreforurl{https://issues.apache.org/jira/browse/MCOMPILER-434}{bug in
  maven-compiler-plugin versions before 3.10.1}: Maven does not report
annotation processor exceptions, even when
the \<-X> command-line argument is passed.


\begin{enumerate}

\item Declare a dependency on the Checker Framework artifacts from
  Maven Central.  Find the
  existing \code{<dependencies>} section (not within
  \code{<dependencyManagement>}, but somewhere else) and add the following new
  \code{<dependency>} items:

\begin{alltt}
  <dependencies>
    ... existing <dependency> items ...

    <!-- Annotations from the Checker Framework: nullness, interning, locking, ... -->
    <dependency>
      <groupId>org.checkerframework</groupId>
      <artifactId>checker-qual</artifactId>
      <version>\ReleaseVersion{}</version>
    </dependency>
  </dependencies>
\end{alltt}

Periodically update to the most recent version, to obtain the
latest bug fixes and new features:
\begin{Verbatim}
  mvn versions:use-latest-versions -Dincludes="org.checkerframework:*"
\end{Verbatim}

\item If using JDK 8, use a Maven property to hold the location of the
  Error Prone \<javac.jar>.
To set the value of these properties automatically, you will use the Maven Dependency plugin.

Add a dependency:

\begin{alltt}
  <dependencies>
    ... existing <dependency> items ...

    <dependency>
      <groupId>com.google.errorprone</groupId>
      <artifactId>javac</artifactId>
      <version>9+181-r4173-1</version>
    </dependency>
  </dependencies>
\end{alltt}

Java 11 and later do not need the \<com.google.errorprone:javac>
dependency.
The need for the \<com.google.errorprone:javac> artifact when running under
JDK 8 is explained in Section~\ref{javac-jdk8}.

Create the property in the \code{properties} section of the POM:

\begin{alltt}
<properties>
  <!-- These properties will be set by the Maven Dependency plugin -->
  <errorProneJavac>$\{com.google.errorprone:javac:jar\}</errorProneJavac>
</properties>
\end{alltt}

Change the reference to the \code{maven-dependency-plugin} within the \code{<plugins>}
section, or add it if it is not present.

\begin{alltt}
  <plugin>
    <!-- This plugin will set properties values using dependency information -->
    <groupId>org.apache.maven.plugins</groupId>
    <artifactId>maven-dependency-plugin</artifactId>
    <executions>
      <execution>
        <goals>
          <goal>properties</goal>
        </goals>
      </execution>
    </executions>
  </plugin>
\end{alltt}

\item Direct the Maven compiler plugin to use the desired checkers by
  creating three new profiles as shown below (the example uses the Nullness
  Checker).  If your POM file does not already use the
  \code{maven-compiler-plugin} plugin, you can use the text below as is.
  If your POM file
  already uses the plugin, copy and edit the first profile to incorporate the
  existing configuration.

%% TODO: Is this comment still accurate?
%              <!-- Without showWarnings and verbose, maven-compiler-plugin may not show output. -->
%              <showWarnings>true</showWarnings>
%              <verbose>true</verbose>

\begin{mysmall}
\begin{alltt}
  <profiles>
    <profile>
      <id>checkerframework</id>
      <activation>
        <jdk>[1.8,)</jdk>
      </activation>
      <build>
        <plugins>
          <plugin>
            <artifactId>maven-compiler-plugin</artifactId>
            <version>3.10.1</version>
            <configuration>
              <fork>true</fork> <!-- Must fork or else JVM arguments are ignored. -->
              <compilerArguments>
                <Xmaxerrs>10000</Xmaxerrs>
                <Xmaxwarns>10000</Xmaxwarns>
              </compilerArguments>
              <annotationProcessorPaths>
                <path>
                  <groupId>org.checkerframework</groupId>
                  <artifactId>checker</artifactId>
                  <version>\ReleaseVersion{}</version>
                </path>
              </annotationProcessorPaths>
              <annotationProcessors>
                <!-- Add all the checkers you want to enable here -->
                <annotationProcessor>org.checkerframework.checker.nullness.NullnessChecker</annotationProcessor>
              </annotationProcessors>
              <compilerArgs>
                <!-- <arg>-Awarns</arg> --> <!-- -Awarns turns type-checking errors into warnings. -->
              </compilerArgs>
            </configuration>
          </plugin>
        </plugins>
      </build>
      <dependencies>
        <dependency>
          <groupId>org.checkerframework</groupId>
          <artifactId>checker</artifactId>
          <version>\ReleaseVersion{}</version>
        </dependency>
      </dependencies>
    </profile>

    <profile>
      <id>checkerframework-jdk8</id>
      <activation>
        <jdk>1.8</jdk>
      </activation>
      <!-- using github.com/google/error-prone-javac is required when running on JDK 8 -->
      <properties>
        <javac.version>9+181-r4173-1</javac.version>
      </properties>
      <dependencies>
        <dependency>
          <groupId>com.google.errorprone</groupId>
          <artifactId>javac</artifactId>
          <version>9+181-r4173-1</version>
        </dependency>
      </dependencies>
      <build>
        <plugins>
          <plugin>
            <groupId>org.apache.maven.plugins</groupId>
            <artifactId>maven-compiler-plugin</artifactId>
            <configuration>
              <fork>true</fork>
              <compilerArgs combine.children="append">
                <arg>-J-Xbootclasspath/p:$\{settings.localRepository\}/com/google/errorprone/javac/$\{javac.version\}/javac-$\{javac.version\}.jar</arg>
              </compilerArgs>
            </configuration>
          </plugin>
        </plugins>
      </build>
    </profile>

    <profile>
      <id>checkerframework-jdk9orlater</id>
      <activation>
        <jdk>[9,)</jdk>
      </activation>
      <build>
        <plugins>
          <plugin>
            <groupId>org.apache.maven.plugins</groupId>
            <artifactId>maven-compiler-plugin</artifactId>
            <configuration>
              <fork>true</fork>
              <compilerArgs combine.children="append">
                <arg>-J--add-exports=jdk.compiler/com.sun.tools.javac.api=ALL-UNNAMED</arg>
                <arg>-J--add-exports=jdk.compiler/com.sun.tools.javac.code=ALL-UNNAMED</arg>
                <arg>-J--add-exports=jdk.compiler/com.sun.tools.javac.file=ALL-UNNAMED</arg>
                <arg>-J--add-exports=jdk.compiler/com.sun.tools.javac.main=ALL-UNNAMED</arg>
                <arg>-J--add-exports=jdk.compiler/com.sun.tools.javac.model=ALL-UNNAMED</arg>
                <arg>-J--add-exports=jdk.compiler/com.sun.tools.javac.processing=ALL-UNNAMED</arg>
                <arg>-J--add-exports=jdk.compiler/com.sun.tools.javac.tree=ALL-UNNAMED</arg>
                <arg>-J--add-exports=jdk.compiler/com.sun.tools.javac.util=ALL-UNNAMED</arg>
                <arg>-J--add-opens=jdk.compiler/com.sun.tools.javac.comp=ALL-UNNAMED</arg>
              </compilerArgs>
            </configuration>
          </plugin>
        </plugins>
      </build>
      <properties>
        <!-- Needed for animal-sniffer-maven-plugin version 1.19 (version 1.20 is fixed). -->
        <animal.sniffer.skip>true</animal.sniffer.skip>
      </properties>
    </profile>
  </profiles>
\end{alltt}
\end{mysmall}

Now, building with Maven will run the checkers during every compilation
that uses JDK 8 or higher.

If you wish to run checkers while compiling your source code but not your
tests, wrap the \code{<configuration>...</configuration>} within

\begin{Verbatim}
<executions>
  <execution>
    <id>default-compile</id>
    ...
  </execution>
</executions>
\end{Verbatim}

To compile without using the Checker Framework, pass
\<-P !checkerframework>
on the Maven command line.
%% Per GitHub user danibs in
%% https://github.com/typetools/checker-framework/issues/5086,
%% this does not seem to work.
% It should also work to pass
% \<-P !checkerframework-jdk9orlater> or \<-P !checkerframework-java8>.

% TODO: Figure out why, and then remove this paragraph.
Warning:  adding

\begin{mysmall}
\begin{alltt}
<compilerArgs>
  ...
  <arg>-verbose</arg>
\end{alltt}
\end{mysmall}

\noindent
may suppress warnings from the stub parser.

\end{enumerate}


\subsectionAndLabel{Maven, with a locally-built version of the Checker Framework}{maven-locally-built}

To use a locally-built version of the Checker Framework, first run:

\begin{alltt}
./gradlew publishToMavenLocal
\end{alltt}

\noindent
Then use the Maven instructions, but modify the version number for the
Checker Framework artifacts.  Instead of \<\ReleaseVersion{}>, use the
version number that is output when you run \<./gradlew version>.


\sectionAndLabel{NetBeans}{netbeans}

There are two approaches to running a checker in NetBeans:  via modifying the
project properties, or via a custom ant target.

Note: The ``compile and save'' action in NetBeans 8.1 automatically
runs the Checker Framework, but this functionality has not yet
been incorporated into NetBeans 8.2. Additionally, JDK annotations
are currently unavailable on NetBeans 8.1 and 8.2.

% NetBeans 8.2 switched to a Java 9 javac, which breaks the Checker Framework as
% annotation processor.
% There is no way to set a bootclasspath for annotation processors, so we can't
% add the JDK annotations.
% Work on a NetBeans plug-in at
% https://github.com/typetools/checker-framework/pull/1592
% could not solve the bootclasspath issue, so we decided to not integrate it.


\subsectionAndLabel{Adding a checker via the Project Properties window}{netbeans-project-properties}

\begin{enumerate}
\item
  Add the Checker Framework libraries to your project's
  library. First, right click on the project in the Projects panel,
  and select ``Properties'' in the drop-down menu. Then, in the
  ``Project Properties'' window, navigate to the ``Libraries'' tab.

\item
  Add \<checker-qual.jar> to the compile-time libraries. To do so,
  select the ``Compile'' tab, click the ``Add JAR/Folder'' button on
  the right and browse to
  add \<\$CHECKERFRAMEWORK/checker/dist/checker-qual.jar>.

\item
  Add \<checker.jar> to the processor-path libraries. To do so, select
  the ``Processor'' tab, click the ``Add JAR/Folder'' button on the
  right and browse to
  add \<\$CHECKERFRAMEWORK/checker/dist/checker.jar>.

\item
  Enable annotation processor underlining in the editor. Go to
  ``Build>Compiling'' and check the box ``Enable Annotation
  Processing,'' and under that, ``Enable Annotation Processing in
  Editor.''

\item
  Add the checker to run, by clicking ``Add'' next to the box labeled
  ``Annotation Processors'' and enter the fully qualified name of the
  checker (for
  example, \<org.checkerframework.checker.nullness.NullnessChecker>)
  and click ``OK'' to add.
\end{enumerate}

The selected checker should be run on the project either on a save (if
Compile on Save is enabled), or when the project is built, and
annotation processor output will appear in the editor.


\subsectionAndLabel{Adding a checker via an ant target}{netbeans-ant-target}

\begin{enumerate}
\item
Set the \code{cfJavac} property:

%BEGIN LATEX
\begin{smaller}
%END LATEX
\begin{Verbatim}
  <property environment="env"/>
  <property name="checkerframework" value="${env.CHECKERFRAMEWORK}" />
  <condition property="cfJavac" value="javac.bat" else="javac">
    <os family="windows" />
  </condition>
  <presetdef name="cf.javac">
    <javac fork="yes" executable="${checkerframework}/checker/bin/${cfJavac}" >
      <compilerarg value="-version"/>
      <compilerarg value="-implicit:class"/>
    </javac>
  </presetdef>
\end{Verbatim}
%BEGIN LATEX
\end{smaller}
%END LATEX

\item
Override the \code{-init-macrodef-javac-with-processors} target to
use \code{cf.javac} instead of \code{javac} and to run the checker.
In this example, a nullness checker is run:

%BEGIN LATEX
\begin{smaller}
%END LATEX
\begin{Verbatim}
  <target depends="-init-ap-cmdline-properties" if="ap.supported.internal"
        name="-init-macrodef-javac-with-processors">
    <echo message = "${checkerframework}"/>
    <macrodef name="javac" uri="http://www.netbeans.org/ns/j2se-project/3">
        <attribute default="${src.dir}" name="srcdir"/>
        <attribute default="${build.classes.dir}" name="destdir"/>
        <attribute default="${javac.classpath}" name="classpath"/>
        <attribute default="${javac.processorpath}" name="processorpath"/>
        <attribute default="${build.generated.sources.dir}/ap-source-output" name="apgeneratedsrcdir"/>
        <attribute default="${includes}" name="includes"/>
        <attribute default="${excludes}" name="excludes"/>
        <attribute default="${javac.debug}" name="debug"/>
        <attribute default="${empty.dir}" name="sourcepath"/>
        <attribute default="${empty.dir}" name="gensrcdir"/>
        <element name="customize" optional="true"/>
        <sequential>
            <property location="${build.dir}/empty" name="empty.dir"/>
            <mkdir dir="${empty.dir}"/>
            <mkdir dir="@{apgeneratedsrcdir}"/>
            <cf.javac debug="@{debug}" deprecation="${javac.deprecation}"
                    destdir="@{destdir}" encoding="${source.encoding}"
                    excludes="@{excludes}" fork="${javac.fork}"
                    includeantruntime="false" includes="@{includes}"
                    source="${javac.source}" sourcepath="@{sourcepath}"
                    srcdir="@{srcdir}" target="${javac.target}"
                    tempdir="${java.io.tmpdir}">
                <src>
                    <dirset dir="@{gensrcdir}" erroronmissingdir="false">
                        <include name="*"/>
                    </dirset>
                </src>
                <classpath>
                    <path path="@{classpath}"/>
                </classpath>
                <compilerarg line="${endorsed.classpath.cmd.line.arg}"/>
                <compilerarg line="${javac.profile.cmd.line.arg}"/>
                <compilerarg line="${javac.compilerargs}"/>
                <compilerarg value="-processorpath"/>
                <compilerarg path="@{processorpath}:${empty.dir}"/>
                <compilerarg line="${ap.processors.internal}"/>
                <compilerarg line="${annotation.processing.processor.options}"/>
                <compilerarg value="-s"/>
                <compilerarg path="@{apgeneratedsrcdir}"/>
                <compilerarg line="${ap.proc.none.internal}"/>
                <compilerarg line="-processor org.checkerframework.checker.nullness.NullnessChecker"/>
                <compilerarg line="-Xmaxerrs 10000"/>
                <compilerarg line="-Xmaxwarns 10000"/>
                <customize/>
            </cf.javac>
        </sequential>
    </macrodef>
  </target>
  <target name="-post-jar">
  </target>
\end{Verbatim}
%BEGIN LATEX
\end{smaller}
%END LATEX

When Build and Clean Project is used, the output of the checker will
now appear in the build console. However, annotation processor output
will not appear in the editor.

\end{enumerate}


\sectionAndLabel{sbt}{sbt}

\href{https://www.scala-sbt.org/}{sbt} is a build tool for
Scala, Java, and more.

Adjust the \<-processor> command-line argument for the processor(s)
you wish to run (see Section~\ref{running}).


\subsectionAndLabel{JDK 8}{sbt-jdk8}

\begin{Verbatim}
javacOptions ++= Seq(
    "-J-Xbootclasspath/p:$CHECKERFRAMEWORK/checker/dist/javac.jar",
    "-cp $CHECKERFRAMEWORK/checker/dist/checker-qual.jar",
    "-processorpath $CHECKERFRAMEWORK/checker/dist/checker.jar",
    "-processor org.checkerframework.checker.nullness.NullnessChecker",
    "-source 8", "-target 8"
  )
\end{Verbatim}


\subsectionAndLabel{JDK 11 and later, for non-modularized code}{sbt-jdk11-nonmodularized-code}

% Keep list of --add-opens and --add-exports in sync with the "maven" and "javac-jdk11-non-modularized" sections,
% the method CheckerMain#getExecArguments, checker-framework/build.gradle in compilerArgsForRunningCFs,
% and in the checker-framework-gradle-plugin, CheckerFrameworkPlugin#applyToProject
\begin{Verbatim}
javacOptions ++= Seq(
    "-J--add-exports=jdk.compiler/com.sun.tools.javac.api=ALL-UNNAMED",
    "-J--add-exports=jdk.compiler/com.sun.tools.javac.code=ALL-UNNAMED",
    "-J--add-exports=jdk.compiler/com.sun.tools.javac.file=ALL-UNNAMED",
    "-J--add-exports=jdk.compiler/com.sun.tools.javac.main=ALL-UNNAMED",
    "-J--add-exports=jdk.compiler/com.sun.tools.javac.model=ALL-UNNAMED",
    "-J--add-exports=jdk.compiler/com.sun.tools.javac.processing=ALL-UNNAMED",
    "-J--add-exports=jdk.compiler/com.sun.tools.javac.tree=ALL-UNNAMED",
    "-J--add-exports=jdk.compiler/com.sun.tools.javac.util=ALL-UNNAMED",
    "-J--add-opens=jdk.compiler/com.sun.tools.javac.comp=ALL-UNNAMED",
    "-processorpath $CHECKERFRAMEWORK/checker/dist/checker.jar",
    "-cp $CHECKERFRAMEWORK/checker/dist/checker-qual.jar",
    "-processor org.checkerframework.checker.nullness.NullnessChecker"
  )
\end{Verbatim}


\subsectionAndLabel{For modularized code}{sbt-modularized-code}

\begin{Verbatim}
For compiling modularized code:
javacOptions ++= Seq(
    "-J--add-exports=jdk.compiler/com.sun.tools.javac.api=ALL-UNNAMED",
    "-J--add-exports=jdk.compiler/com.sun.tools.javac.code=ALL-UNNAMED",
    "-J--add-exports=jdk.compiler/com.sun.tools.javac.file=ALL-UNNAMED",
    "-J--add-exports=jdk.compiler/com.sun.tools.javac.main=ALL-UNNAMED",
    "-J--add-exports=jdk.compiler/com.sun.tools.javac.model=ALL-UNNAMED",
    "-J--add-exports=jdk.compiler/com.sun.tools.javac.processing=ALL-UNNAMED",
    "-J--add-exports=jdk.compiler/com.sun.tools.javac.tree=ALL-UNNAMED",
    "-J--add-exports=jdk.compiler/com.sun.tools.javac.util=ALL-UNNAMED",
    "-J--add-opens=jdk.compiler/com.sun.tools.javac.comp=ALL-UNNAMED",
    "-processorpath $CHECKERFRAMEWORK/checker/dist/checker.jar",
    "--module-path $CHECKERFRAMEWORK/checker/dist/checker-qual.jar",
    "-processor org.checkerframework.checker.nullness.NullnessChecker"
  )
\end{Verbatim}


\sectionAndLabel{tIDE}{tide}

\begin{sloppypar}
tIDE, an open-source Java IDE, supports the Checker Framework.
You can download it from \myurl{https://sourceforge.net/projects/tide/}.
\end{sloppypar}


\sectionAndLabel{Type inference tools}{type-inference-varieties}

A type inference tool infers type annotations for a program's method
signatures and fields, so that the programmer does not need to manually
annotate the program's source code.  Section~\ref{type-inference-tools}
lists type inference tools.



% LocalWords:  jsr plugin Warski xml buildfile tIDE java Awarns pom lifecycle
% LocalWords:  IntelliJ Maia newdir classpath Unconfuse nullness Gradle cp
% LocalWords:  compilerArgs Xbootclasspath mvn sbt runtime jdk11
% LocalWords:  plugins proc procOnly DirectoryScanner setIncludes groupId
% LocalWords:  setExcludes checkerFrameworkVersion javacParams javaParams
% LocalWords:  artifactId quals failOnError ejc CHECKERFRAMEWORK env jdk
% LocalWords:  javacheck checkerframework MavenExample org arg typecheck
% LocalWords:  AoutputArgsToFile qual jdk8 Unhandled dex SDK
% LocalWords:  annotationProcessors annotationProcessor JavaCompile Ctrl
% LocalWords:  targetJavaVersion GradleExamples gradle JavaVersion lombok
% LocalWords:  systemPath artifactID MacOS eclipsec getter getRole setRole
% LocalWords:  config copyableAnnotations COPYABLE localRepository init FQ
% LocalWords:  compilerArguments Xmaxerrs Xmaxwarns netbeans macrodef LSP
% LocalWords:  annotationProcessorPaths checkTypes OracleJDK java8 java11
% LocalWords:  bootclasspath processorpath intellij typechecking postpass
% LocalWords:  Delombok r4173 pathnames HandlerUtil errorProneJavac
% LocalWords:  uncomment
