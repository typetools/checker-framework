\htmlhr
\chapterAndLabel{Building an accumulation checker}{accumulation-checker}

%% This chapter should appear after the "creating a checker" chapter, or perhaps as part of it,
%% once accumulation support is complete.

This chapter describes how to build a particular kind of checker---an accumulation
checker. If you want to \emph{use} an existing checker, you do not need to read this chapter.

An \emph{accumulation analysis} is a program analysis where the
analysis abstraction is a monotonically increasing set, and some operation is legal
only when the set is large enough---that is, the estimate has accumulated
sufficiently many items.
Accumulation analysis is a special case of typestate analysis in which
(1) the order in which operations are performed does not affect what is subsequently legal,
and (2) the accumulation does not add restrictions; that is, as
more operations are performed, more operations become legal.

Unlike a traditional typestate analysis, an accumulation analysis does
not require an alias analysis for soundness. It can therefore be implemented
as a flow-sensitive type system.

The Checker Framework contains a generic accumulation analysis that can
be extended to implement any particular accumulation analysis. We refer to
such an extension as an "accumulation checker" in this chapter.

Before reading the rest of this chapter, you should
read the section on creating a checker (Chapter~\ref{creating-a-checker}).
The rest of this chapter assumes you have done so.

\paragraphAndLabel{Defining type qualifiers}{accumulation-qualifiers}

The first step in building an accumulation checker is to decide what it is
your checker will accumulate, and how to represent it. The Checker Framework's
support for accumulation analysis requires you to accumulate a string representation
of whatever you are accumulating. For example, when accumulating which methods have
been called on an object, you might choose to accumulate method names.

Define a type qualifier that has a single argument: a \<String[]> named \<value>.
Choose a descriptive name for the qualifier that suggests what is being accumulated.
An example of such a qualifier can be found in the Checker Framework's tests, at
\<$CHECKERFRAMEWORK/framework/src/test/java/testaccumulation/qual/TestAccumulation.java>.

Also define top and bottom types, analogous to
\<$CHECKERFRAMEWORK/framework/src/test/java/testaccumulation/qual/TestAccumulationTop.java>
and
\<$CHECKERFRAMEWORK/framework/src/test/java/testaccumulation/qual/TestAccumulationBottom.java>.

\paragraphAndLabel{Setting up the checker}{accumulation-setup}

Define a new class that extends \refclass{common/accumulation}{AccumulationChecker}.
It does not need any content.

Define a new class that extends \refclass{common/accumulation}{AccumulationAnnotatedTypeFactory}.
You must create a new constructor whose only argument is a \refclass{common/basetype}{BaseTypeChecker}.
Your constructor should call the \<super> constructor defined in
\refclass{common/accumulation}{AccumulationAnnotatedTypeFactory}. That constructor takes four arguments:
\begin{itemize}
\item the \refclass{common/basetype}{BaseTypeChecker} argument to your constructor,
\item the class that defines your accumulation annotation,
\item the class that defines your top annotation, and
\item the class that defines your bottom annotation.
\end{itemize}

\paragraphAndLabel{Adding accumulation logic}{accumulation-accumulating}

Define a class that extends \refclass{common/accumulation}{AccumulationTransfer}.
In your subclass, you should call the
\refmethodterse{common/accumulation}{AccumulationTransfer}{accumulate}{-org.checkerframework.dataflow.cfg.node.Node-org.checkerframework.dataflow.analysis.TransferResult-java.lang.String...-}
or
\refmethodterse{common/accumulation}{AccumulationTransfer}{accumulate}{-org.checkerframework.dataflow.cfg.node.MethodInvocationNode-org.checkerframework.dataflow.analysis.TransferResult-java.lang.String...-}
methods when your checker should update its estimate of what has been accumulated.
To do so, you will need to override one or more methods of \refclass{framework/flow}{CFAbstractTransfer},
as you would when defining a typical transfer. The \<accumulate> methods will handle updating the store,
etc., for you.

For example, when accumulating the names of methods called, you would override
\refmethod{framework/flow}{CFAbstractTransfer}{visitMethodInvocation}{-org.checkerframework.dataflow.cfg.node.MethodInvocationNode-org.checkerframework.dataflow.analysis.TransferInput-},
call \<super> to get a \<TransferResult>, compute the method name from the \<MethodInvocationNode>,
and then call \<accumulate>.
