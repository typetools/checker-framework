\htmlhr
\chapterAndLabel{Type inference}{type-inference}

There are two different tasks that are commonly called ``type inference'':
\begin{enumerate}
\item
  \emph{Local} type inference during type-checking:
  The type-checker fills in an appropriate type where the programmer didn't
  write one, but does not change the source code.
  See Section~\ref{type-inference-refinement}.
\item
  \emph{Global} type inference as a separate step:
  A type inference tool finds a set of type qualifiers that are consistent
   with the source code.
  See Section~\ref{global-type-inference}.
\end{enumerate}

Each variety has its own advantages, discussed below.
Advantages of \emph{all} varieties of type inference include:
\begin{itemize}
\item
  Less work for the programmer.
\item
  The tool chooses the most general type, whereas a programmer might
  accidentally write a more-specific, less-generally-useful annotation.
\end{itemize}

All varieties of type inference share a common disadvantage: increased computational
cost. Because type-checking is modular and local type inference is restricted to a single method body,
the cost of local inference is small. Global type inference might need to consider the entire program to infer a single
type qualifier, so it is usually significantly more expensive than type-checking (sometimes many times more expensive).

\sectionAndLabel{Local type inference during type-checking}{type-inference-refinement}

During type-checking, if certain variables have no programmer-written type qualifier, the
type-checker determines whether there is some type qualifier that would
permit the program to type-check.  If so, the type-checker uses that type
qualifier, but does not change the source code.  Each time the
type-checker runs, it re-infers the type qualifier for that variable.  If
no type qualifier exists that permits the program to type-check, the
type-checker issues a warning.

Local type inference is built into the Checker Framework.
Every checker automatically uses it.  As a result, a programmer typically
does not have to write any qualifiers inside the body of a method
(except occasionally on type arguments).
However, it primarily
works within a method, not across method boundaries.
The source code must already contain annotations for method
signatures (arguments and return values) and fields.

Advantages of this variety of type inference include:
\begin{itemize}
\item
  If the type qualifier is obvious to the programmer, then omitting it
  can reduce annotation clutter in the program.
\item
  If the code changes, then there is no old annotation that
  might need to be updated.
\item
  Within-method type inference occurs automatically.
  The programmer doesn't have to do anything to take advantage of it.
\end{itemize}

For more details about local type inference during type-checking, also
known as ``flow-sensitive type refinement'', see
Section~\ref{type-refinement}.


\sectionAndLabel{Global type inference}{global-type-inference}

As a separate step before type-checking, a type inference tool takes the
program as input, and outputs a set of type qualifiers that would
make the program type-check.  (If no such set exists, for example because
the program is not type-correct, then the inference tool does its best but
makes no guarantees.)
These qualifiers can be inserted into the source code or the
class file, or stored separately in an annotation file.
They can be viewed and adjusted by the programmer, and can
be used by tools such as the type-checker.

Advantages of this variety of type inference include:
\begin{itemize}
\item
  The inference may be more precise by taking account of the entire program
  rather than just reasoning one method at a time.
\item
  The inference process is fully automatic, so it can save time for programmers who
  would otherwise have to write many annotations.
\item
  The inference process does not require the user to understand the source code
  of the target program unless the inference tool cannot find a consistent set of
  type qualifiers. For type systems that require few type qualifiers and rarely issue
  errors (e.g. high-precision type systems that detect rare security errors), this sort
  of inference can enable a single user scan many potentially-erroneous codebases.
\end{itemize}

Inserting the inferred annotations into the program source code creates documentation in the form of type
qualifiers, which can aid programmer understanding and may make
type-checking warnings more comprehensible.
Storing annotations in side-files is more desirable if the program's source code cannot
be modified for some reason, or if the typechecking is "one-off": that is, typechecking will
be done once and its results will be evaluated, but it will not be done repeatedly.

\subsectionAndLabel{Type inference tools}{type-inference-tools}

This section lists tools that take a program and output a set of
annotations for it.
It first lists tools that work only for a single type system (but may do a
more accurate job for that type system)
then lists general tools that work for any type system.


\subsubsectionAndLabel{Type inference for specific type systems}{type-inference-tools-specialized}

Section~\ref{nullness-inference} lists several tools that infer
annotations for the Nullness Checker.

If you run the Checker Framework with the \<-AsuggestPureMethods>
command-line option, it will suggest methods that can be marked as
\<@SideEffectFree>, \<@Deterministic>, or \<@Pure>; see
Section~\ref{type-refinement-purity}.


\subsubsectionAndLabel{Type inference for any type system}{type-inference-tools-general}

``Whole program inference'', or WPI, utilizes the \<-Ainfer=\emph{outputformat}>
command-line option.  See Section~\ref{whole-program-inference}.

\href{https://github.com/opprop/checker-framework-inference}{``Checker
  Framework Inference''}, or CFI, is a type inference framework built on
the Checker Framework.  You need to slightly rewrite your type system to
work with CFI\@.  The CFI repository contains rewritten versions of some of
the type systems that are distributed with the Checker Framework.

\href{https://github.com/reprogrammer/cascade/}{Cascade}~\cite{VakilianPEJ2014}
is an Eclipse plugin that implements interactive type qualifier inference.
Cascade is interactive rather than fully-automated:  it makes it easier for
a developer to insert annotations.
Cascade starts with an unannotated program and runs a type-checker.  For each
warning it suggests multiple fixes, the developer chooses a fix, and
Cascade applies it.  Cascade works with any checker built on the Checker
Framework.
You can find installation instructions and a video tutorial at \url{https://github.com/reprogrammer/cascade}.
% See last commit at https://github.com/reprogrammer/cascade/commits/master .
Cascade was last updated in November 2014, so it might or might not work for you.


\sectionAndLabel{Whole-program inference}{whole-program-inference}

Whole-program inference is an interprocedural inference that
infers types for fields, method parameters, and method return types that do not
have a user-written qualifier (for the given type system).
The inferred type qualifiers are output into annotation files.
The inferred type is the most specific type that is compatible with all the
uses in the program.  For example, the inferred type for a field is the
least upper bound of the types of all the expressions that are assigned
into the field.

Whole-program inference differs from type refinement (Section~\ref{type-refinement})
in three ways.  First, type refinement only works within a method body.
Second, type refinement always
refines the current type, regardless of whether the value already has an
annotation in the source code.
Third, whole-program inference can infer a subtype
or a supertype of the default type, by contrast with type refinement which
always refines the current type to a subtype.

There are three scripts that you can use to run whole-program inference.
Each has advantages and disadvantages, discussed below:

\begin{itemize}
    \item
    To run whole-program inference on a single project without modifying its source code,
    use the \<wpi.sh> script (Section~\ref{wpi-one}). This script can automatically understand
    many Ant, Maven, and Gradle build files, so it requires little manual configuration.

    \item
    To run whole-program inference on many projects without modifying their source code
    (say, when running it on projects from GitHub), use the \<wpi-many.sh> script (Section~\ref{wpi-many}).
    This script can understand the same build files as \<wpi.sh>.

    \item
    If you want to insert the inferred annotations directly into a single
    project's source code, use the \<infer-and-annotate.sh> script (Section~\ref{wpi-insert}).
    This script requires significantly more manual configuration than the other two
    scripts because it does not automatically infer how to build your project.
\end{itemize}

All WPI-related scripts appear in the \<checker/bin/> directory.

\subsectionAndLabel{Shared requirements for whole-program inference scripts that do not insert annotations into source code}{wpi-shared-requirements}

The requirements to run \<wpi.sh> and \<wpi-many.sh> are the same:

\begin{itemize}
\item The project on which inference is run must contain an Ant, Gradle,
  or Maven buildfile that compiles the project.
\item \verb|JAVA8_HOME| environment variable must point to a Java 8 JDK.
\item \verb|JAVA11_HOME| environment variable must point to a Java 11 JDK.
\item \<CHECKERFRAMEWORK> environment variable must point to a built copy of the Checker Framework.
\item Other dependencies:
  ant,
  awk,
  curl,
  git,
  gradle,
  mvn,
  python2.7 (for dljc),
  wget.

  Python2.7 modules:
  subprocess32.
\end{itemize}

\subsectionAndLabel{Running whole-program inference on a single project}{wpi-one}

The requirements to run \<wpi.sh> and \<wpi-many.sh> are the same. See Section~\ref{wpi-shared-requirements}
for the list of requirements.

A typical invocation of \<wpi.sh> is

\begin{Verbatim}
  wpi.sh -d . -- --checker nullness
\end{Verbatim}

The result is a set of log files placed in the \<dljc-out> folder of the
target project. The results of type-checking with each candidate set of
annotations will be concatenated into the file \<dljc-out/wpi.log>.

The full syntax for invoking \<wpi.sh> is

\begin{Verbatim}
  wpi.sh [-d PROJECTDIR] [-t TIMEOUT] -- [DLJC-ARGS]
\end{Verbatim}

Arguments in square brackets are optional.
Here is an explanation of the arguments:

\begin{description}
\item[-d PROJECTDIR]
  The top-level directory of the project.  It must contain an Ant, Gradle,
  or Maven buildfile. The default is the current working directory.

\item[-t TIMEOUT]
  The timeout for running the checker, in seconds

\item[DLJC-ARGS]
  Arguments that are passed directly to
  \ahref{https://github.com/kelloggm/do-like-javac}{do-like-javac}'s
  \<dljc> program without
  modification.  One argument is required: \<--checker>, which indicates
  what type-checker to run (use the fully-qualified name of the checker).
  The \ahref{https://github.com/kelloggm/do-like-javac/README.md}{documentation of do-like-javac}
  describes the other commands that its WPI tool supports.
\end{description}

You may need to wait a few minutes for the command to complete.

\subsectionAndLabel{Running whole-program inference on many projects}{wpi-many}

The requirements to run \<wpi.sh> and \<wpi-many.sh> are the same. See Section~\ref{wpi-shared-requirements}
for the list of requirements.

To run an experiment on many projects:
\begin{enumerate}
\item Use \<query-github.sh> to search GitHub for candidate repositories.
File \<securerandom.query> is an example query, and file
\<securerandom.list>
was created by running \<query-github.sh securerandom.query 100>. If you do
not want to use GitHub, construct a file yourself that matches the format of
the file \<securerandom.list>.

\item Use \<wpi-many.sh> to run whole-program inference on every
Gradle or Maven project in a list of (GitHub repository URL, git hash)
pairs.
\begin{itemize}
\item If you are using a checker that is distributed with the Checker
Framework, use wpi-many.sh directly.
\item If you are using a checker that is not distributed with the Checker
Framework (also known as a "custom checker"), file
\<custom-checker-example.sh> is a no-arguments
script that serves as an example of how to use \<wpi-many.sh>.
\end{itemize}

Log files are copied into a results directory.
For a failed run, the log file indicates the reason that WPI could not
be run to completion on the project.
For a successful run, the log file indicates whether the project was verified
(i.e. no errors were reported), or whether the checker issued warnings
(which might be true positive or false positive warnings).

\item Use \<wpi-summary.sh> to summarize the logs in the output results directory.
Use its output to guide your analysis of the results of running \<wpi-many.sh>:
you should manually examine the log files for the projects that appear in the
"results available" list it produces. This list is the list of every project
that the script was able to successfully run WPI on.

\item (Optional) Fork repositories and make changes (e.g., add annotations or fix bugs).
Modify the input file for wpi-many.sh to remove the line for the original repository,
but add a new line that indicates the location of both your
fork and the original repository.
Then, re-run your experiments, supplying the \<-u "\$yourGithubId"> option to \<wpi-many.sh>.
\<wpi-many.sh> will perform inference on your forked version rather than
the original.

\end{enumerate}

The \<wpi-many.sh> script takes the following command-line arguments, of
which \<-o> and \<-i> are mandatory:

\begin{description}
\item[-o outdir]
  run the experiment in the ./outdir directory, and place the results in
  the ./outdir-results directory. Both will be created if they do not
  exist.

\item[-i infile]
  Read the list of repositories to use from the file infile. Each line
  should have 2 or 3 elements, separated by whitespace:
  \begin{enumerate}
  \item
    The URL of the git repository on GitHub. The URL must be of the form:
    https://github.com/username/repository .  The script is reliant on the
    number of slashes, so excluding https:// is an error.
  \item The commit hash to use.
  \item
    If the repository's owner is the user specified by the -u flag, the
    original (upstream) GitHub repository.  Its only use is to be made an
    upstream named "unannotated".
  \end{enumerate}

\item[-t timeout]
  The timeout for running the checker on each project, in seconds.

\item[-u user]
  The GitHub owner for repositories that have been forked and
  modified. These repositories must have a third entry in the infile
  indicating their origin. Default is "\$USER".

\end{description}

\subsectionAndLabel{Whole-program inference that inserts annotations into source code}{wpi-insert}

\begin{sloppypar}
To use this version of whole-program inference, make sure that
\<insert-annotations-to-source>, from the Annotation File Utilities project,
is on your path (for example, its directory is in the \<\$PATH> environment variable).
Then, run the script \<checker-framework/checker/bin/infer-and-annotate.sh>.
Its command-line arguments are:
\end{sloppypar}

\begin{enumerate}
\item Optional: Command-line arguments to
  \href{https://checkerframework.org/annotation-file-utilities/#insert-annotations-to-source}{\<insert-annotations-to-source>}.
\item Processor's name.
\item Target program's classpath.  This argument is required; pass "" if it
  is empty.
\item Optional: Extra processor arguments which will be passed to the checker, if any.
  You may supply any number of such arguments, or none.  Each such argument
  must start with a hyphen.
\item Optional: Paths to \<.jaif> files used as input in the inference
    process.
\item Paths to \<.java> files in the program.
\end{enumerate}

% TODO: Change the example project that is being annotated, since plume-lib is now deprecated.
For example, to add annotations to the \<plume-lib> project:
\begin{Verbatim}
git clone https://github.com/mernst/plume-lib.git
cd plume-lib
make jar
$CHECKERFRAMEWORK/checker/bin/infer-and-annotate.sh \
    "LockChecker,NullnessChecker" java/plume.jar:java/lib/junit-4.12.jar:$JAVA_HOME/lib/tools.jar \
    `find java/src/plume/ -name "*.java"`
# View the results
git diff
\end{Verbatim}

You may need to wait a few minutes for the command to complete.
You can ignore warnings that the command outputs while it tries different
annotations in your code.

It is recommended that you run \<infer-and-annotate.sh> on a copy of your
code, so that you can see what changes it made and so that it does not
change your only copy.  One way to do this is to work in a clone of your
repository that has no uncommitted changes.

\subsectionAndLabel{Whole-program inference ignores some code}{whole-program-inference-ignores-some-code}

Whole-program inference ignores code within the scope of a
\<@SuppressWarnings> annotation with an appropriate key
(Section~\ref{suppresswarnings-annotation}).  In particular, uses within
the scope do not contribute to the inferred type, and declarations within
the scope are not changed.  You should remove \<@SuppressWarnings> annotations
from the class declaration of any class you wish to infer types for.

As noted below, whole-program inference requires invocations of your code, or
assignments to your methods, to generalize from.  If a field is set via
reflection (such as via injection), then whole-program inference would produce
an inaccurate result.  There are two ways to make whole-program inference
ignore such a field.
%
(1)
You probably have an annotation such as
\javaeejavadocanno{javax/inject/Inject.html}{Inject}
or
\href{https://types.cs.washington.edu/plume-lib/api/plume/Option.html}{\<@Option>}
that indicates such fields.  Meta-annotate the declaration of the \<Inject>
or \<Option> annotation with
\refqualclass{framework/qual}{IgnoreInWholeProgramInference}.
%
(2)
Annotate the field to be ignored with
\refqualclass{framework/qual}{IgnoreInWholeProgramInference}.

Whole-program inference, for a type-checker other than the Nullness Checker,
ignores (pseudo-)assignments where the right-hand-side is the \<null> literal.


\subsectionAndLabel{Manually checking whole-program inference results}{whole-program-inference-manual-checking}

As with any type inference tool, it is a good idea to manually examine the
results.

\begin{itemize}
\item
Whole-program inference can produce undesired results when your code has
non-representative or erroneous calls to a particular method or assignments to a
particular field, as explained below.
This is especially noticeable when the arguments or assignments are literals.

\item
If an annotation is inferred for a \emph{use} of type variables;
it might be more appropriate for you to move those annotations
to the corresponding upper bounds of the type variable \emph{declaration}.

\end{itemize}


\subsubsectionAndLabel{Poor whole-program inference results due to non-representative uses}{whole-program-inference-non-representative-uses}

Whole-program inference determines a method parameter's type
annotation based on what arguments are passed to the method, but not on how the
parameter is used within the method body.

\begin{itemize}
\item
If the program contains erroneous calls, the
inferred annotations may reflect those errors.

Suppose you intend method \<m2> to be called with
non-null arguments, but your program contains an error and one of the calls
to \<m2> passes \<null> as the argument.  Then the tool will infer that
\<m2>'s parameter has \<@Nullable> type.
You should correct the bug and re-run inference.

\item
If the program uses (say) a method in a limited way, then the inferred
annotations will be legal for the program as
currently written but may not be as general as possible and may not
accommodate future program changes.

Here are some examples:

\begin{itemize}
\item
Suppose that your program currently calls
method \<m1> only with non-null
arguments.  The tool will infer that \<m1>'s parameter has
\<@NonNull> type.  If you had intended the method to be able to
take \<null> as an argument and you later add such a call, the type-checker
will issue a warning because the automatically-inserted \<@NonNull>
annotation is inconsistent with the new call.

\item
If your program (or test suite) passes only \<null> as an argument, the
inferred type will be the bottom type, such as \<@GuardedByBottom>.

\item
It is common for whole-program inference to infer
\<@Interned> and \<@Regex> annotations on String variables for which the
analyzed code only uses a constant string.

\end{itemize}

In each case, you can correct the inferred results manually, or you can
add tests that pass additional values then re-run inference.

\end{itemize}


\subsectionAndLabel{How whole-program inference works}{how-whole-program-inference-works}

This section explains how the \<wpi.sh> and \<infer-and-annotate.sh> scripts work.  If you
merely want to run the scripts and you are not encountering trouble, you can
skip this section.

Each script repeatedly the checker with an \<-Ainfer=> command-line option to infer
types for fields and method signatures.  The output of this step
is a \<.jaif> (for \<infer-and-annotate.sh>) or stub (for \<wpi.sh>) file that records the inferred types.
Each script adds the inferred annotation to the next run, so that the checker takes them into
account (and checks them). \<wpi.sh> does this using the \<-AmergeStubsWithSource> command-line
option to the Checker Framework; \<infer-and-annotate.sh> insers the inferred annotations in the program using the
Annotation File Utilities(\myurl{https://checkerframework.org/annotation-file-utilities/}).

The process halts when there are no more changes to the inference results,
that is, the \<.jaif> or \<.astub> files are unchanged between two runs.  On each
iteration through the process, there may be new annotations in the \<.jaif> or \<.astub>
files, and some type-checking errors may be eliminated (though others might
be introduced).

When the type-checker is run on the program with the final annotations
inserted, there might still be errors.  This may be because the tool did
not infer enough annotations, or because your program cannot typecheck
(either because contains a defect, or because it contains subtle code that
is beyond the capabilities of the type system).
However, each of the inferred annotations is sound, and this reduces your
manual effort in annotating the program.

The iterative process is required because type-checking is modular:  it
processes each class and each method only once, independently.  Modularity
enables you to run type-checking on only part of your program, and it makes
type-checking fast.  However, it has some disadvantages:
\begin{itemize}
\item
  The first run of the type-checker cannot take advantage of whole-program
  inference results because whole-program inference is only complete at the
  end of type-checking, and modular type-checking does not revisit any
  already-processed classes.
\item
  Revisiting an
  already-processed class may result in a better estimate.
\end{itemize}



%%  LocalWords:  Ainfer java jaif plugin classpath m2 m1 multi javax
%%  LocalWords:  AsuggestPureMethods CHECKERFRAMEWORK GuardedByBottom
%%  LocalWords:  IgnoreInWholeProgramInference
