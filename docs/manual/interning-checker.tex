\htmlhr
\chapter{Interning Checker\label{interning-checker}}

If the Interning Checker issues no errors for a given program, then all
reference equality tests (i.e., all uses of ``\code{==}'') are proper;
that is,
\code{==} is not misused where \code{equals()} should have been used instead.

Interning is a design pattern in which the same object is used whenever two
different objects would be considered equal.  Interning is also known as
canonicalization or hash-consing, and it is related to the flyweight design
pattern.
Interning has two benefits:  it can save memory, and it can speed up testing for
equality by permitting use of \code{==}.

The Interning Checker prevents two types of errors in your code.  First,
\code{==} should be used
only on interned values; using \code{==} on
non-interned values can result in subtle bugs.  For example:

\begin{Verbatim}
  Integer x = new Integer(22);
  Integer y = new Integer(22);
  System.out.println(x == y);  // prints false!
\end{Verbatim}

\noindent
The Interning Checker helps programmers to prevent such bugs.
Second,
the Interning Checker also helps to prevent performance problems that result
from failure to use interning.
(See Section~\ref{checker-guarantees} for caveats to the checker's guarantees.)

Interning is such an important design pattern that Java builds it in for
these types: \<String>, \<Boolean>, \<Byte>, \<Character>, \<Integer>,
\<Short>.  Every string literal in the program is guaranteed to be interned
(\href{https://docs.oracle.com/javase/specs/jls/se8/html/jls-3.html#jls-3.10.5}{JLS
  \S3.10.5}), and the
\sunjavadoc{java/lang/String.html\#intern--}{String.intern()} method
performs interning for strings that are computed at run time.
The \<valueOf> methods in wrapper classes always (\<Boolean>, \<Byte>) or
sometimes (\<Character>, \<Integer>, \<Short>) return an interned result
(\href{https://docs.oracle.com/javase/specs/jls/se8/html/jls-5.html#jls-5.1.7}{JLS \S5.1.7}).
Users can also write their own interning methods for other types.

It is a proper optimization to use \code{==}, rather than \code{equals()},
whenever the comparison is guaranteed to produce the same result --- that
is, whenever the comparison is never provided with two different objects
for which \code{equals()} would return true.  Here are three reasons that
this property could hold:

\begin{enumerate}
\item
  Interning.  A factory method ensures that, globally, no two different
  interned objects are \code{equals()} to one another.  (In some cases
  other, non-interned objects of the class might be \code{equals()} to one
  another; in other cases, every object of the class is interned.)
  Interned objects should always be immutable.
\item
  Global control flow.  The program's control flow is such that the
  constructor for class $C$ is called a limited number of times, and with
  specific values that ensure the results are not \code{equals()} to one
  another.  Objects of class $C$ can always be compared with \code{==}.
  Such objects may be mutable or immutable.
\item
  Local control flow.  Even though not all objects of the given type may be
  compared with \code{==}, the specific objects that can reach a given
  comparison may be.  For example, suppose that an array contains no
  duplicates.  Then testing to find the index of a given element that is
  known to be in the array can use \code{==}.
\end{enumerate}

To eliminate Interning Checker errors, you will need to annotate the
declarations of any expression used as an argument to \code{==}.
Thus, the Interning Checker
could also have been called the Reference Equality Checker.  In the
future, the checker will include annotations that target the non-interning
cases above, but for now you need to use \<@Interned>, \<@UsesObjectEquals>
(which handles a surprising number of cases), and/or
\<@SuppressWarnings>.

\begin{sloppypar}
To run the Interning Checker, supply the
\code{-processor org.checkerframework.checker.interning.InterningChecker}
command-line option to javac.  For examples, see Section~\ref{interning-example}.
\end{sloppypar}


\section{Interning annotations\label{interning-annotations}}

These qualifiers are part of the Interning type system:

\begin{description}

\item[\refqualclass{checker/interning/qual}{Interned}]
  indicates a type that includes only interned values (no non-interned
  values).

\item[\refqualclass{checker/interning/qual}{UnknownInterned}]
  indicates a type whose values might or might not be interned.
  It is used internally by the type system and is not written by programmers.

\item[\refqualclass{checker/interning/qual}{PolyInterned}]
  indicates qualifier polymorphism (see
  Section~\ref{qualifier-polymorphism}).

\item[\refqualclass{checker/interning/qual}{UsesObjectEquals}]
  is a class (not type) annotation that indicates that this class's
  \<equals> method is the same as that of \<Object>.  In other words,
  neither this class nor any of its superclasses overrides the \<equals>
  method.  Since \<Object.equals> uses reference equality, this means that
  for such a class, \<==> and \<equals> are equivalent, and so the
  Interning Checker does not issue errors or warnings for either one.

\end{description}


\section{Annotating your code with \code{@Interned}\label{annotating-with-interned}}

\begin{figure}
\includeimage{interning}{2.5cm}
\caption{Type hierarchy for the Interning type system.}
\label{fig-interning-hierarchy}
\end{figure}

In order to perform checking, you must annotate your code with the \refqualclass{checker/interning/qual}{Interned}
type annotation, which indicates a type for the canonical representation of an
object:

%BEGIN LATEX
\begin{smaller}
%END LATEX
\begin{Verbatim}
            String s1 = ...;  // type is (uninterned) "String"
  @Interned String s2 = ...;  // Java type is "String", but checker treats it as "@Interned String"
\end{Verbatim}
%BEGIN LATEX
\end{smaller}
%END LATEX

The type system enforced by the checker plugin ensures that only interned
values can be assigned to \code{s2}.

To specify that \emph{all} objects of a given type are interned, annotate the
class declaration:

\begin{Verbatim}
  public @Interned class MyInternedClass { ... }
\end{Verbatim}

This is equivalent to annotating every use of \code{MyInternedClass}, in a
declaration or elsewhere.  For example, \code{enum} classes are implicitly
so annotated.


\subsection{Implicit qualifiers\label{interning-implicit-qualifiers}}

As described in Section~\ref{effective-qualifier}, the Interning Checker
adds implicit qualifiers, reducing the number of annotations that must
appear in your code.
For example, String literals and the \<null> literal are always considered interned, and
object creation expressions (using \code{new}) are never considered
\refqualclass{checker/interning/qual}{Interned} unless they are annotated as such, as in

%BEGIN LATEX
\begin{smaller}
%END LATEX
\begin{Verbatim}
@Interned Double internedDoubleZero = new @Interned Double(0); // canonical representation for Double zero
\end{Verbatim}
%BEGIN LATEX
\end{smaller}
%END LATEX

For a complete description of all implicit interning qualifiers, see the
Javadoc for \refclass{checker/interning}{InterningAnnotatedTypeFactory}.


\section{What the Interning Checker checks\label{interning-checks}}

Objects of an \refqualclass{checker/interning/qual}{Interned} type may be safely compared using the ``\code{==}''
operator.

The checker issues an error in two cases:

\begin{enumerate}

\item
  When a reference (in)equality operator (``\code{==}'' or ``\code{!=}'')
  has an operand of non-\refqualclass{checker/interning/qual}{Interned} type.

\item
  When a non-\refqualclass{checker/interning/qual}{Interned} type is used where an \refqualclass{checker/interning/qual}{Interned} type
  is expected.

\end{enumerate}

This example shows both sorts of problems:

\begin{Verbatim}
            Date  date;
  @Interned Date idate;
  ...
  if (date == idate) { ... }  // error: reference equality test is unsafe
  idate = date;               // error: idate's referent may no longer be interned
\end{Verbatim}

\label{lint-dotequals}

The checker also issues a warning when \code{.equals} is used where
\code{==} could be safely used.  You can disable this behavior via the
javac \code{-Alint} command-line option, like so: \code{-Alint=-dotequals}.

For a complete description of all checks performed by
  the checker, see the Javadoc for
  \refclass{checker/interning}{InterningVisitor}.

\label{checking-class}
You can also restrict which types the checker should examine and type-check,
using the \code{-Acheckclass} option.  For example, to find only the
interning errors related to uses of \code{String}, you can pass
\code{-Acheckclass=java.lang.String}.  The Interning Checker always checks all
subclasses and superclasses of the given class.


\subsection{Limitations of the Interning Checker\label{interning-limitations}}

% There is no point to linking to the Javadoc for the valueOf methods,
% which don't discuss interning.

The Interning Checker conservatively assumes that the \<Character>, \<Integer>,
and \<Short> \<valueOf> methods return a non-interned value.  In fact, these
methods sometimes return an interned value and sometimes a non-interned
value, depending on the run-time argument (\href{https://docs.oracle.com/javase/specs/jls/se8/html/jls-5.html#jls-5.1.7}{JLS
\S5.1.7}).  If you know that the run-time argument to \<valueOf> implies that
the result is interned, then you will need to suppress an error.  (An
alternative would be to enhance the Interning Checker to estimate the upper
and lower bounds on char, int, and short values so that it can more
precisely determine whether the result of a given \<valueOf> call is
interned.)



\section{Examples\label{interning-example}}

To try the Interning Checker on a source file that uses the \refqualclass{checker/interning/qual}{Interned} qualifier,
use the following command (where \code{javac} is the Checker Framework compiler that
is distributed with the Checker Framework):

\begin{mysmall}
\begin{Verbatim}
  javac -processor org.checkerframework.checker.interning.InterningChecker docs/examples/InterningExample.java
\end{Verbatim}
\end{mysmall}

\noindent
Compilation will complete without errors or warnings.

To see the checker warn about incorrect usage of annotations, use the following
command:

\begin{mysmall}
\begin{Verbatim}
  javac -processor org.checkerframework.checker.interning.InterningChecker docs/examples/InterningExampleWithWarnings.java
\end{Verbatim}
\end{mysmall}

\noindent
The compiler will issue an error regarding violation of the semantics of
\refqualclass{checker/interning/qual}{Interned}.
% in the \code{InterningExampleWithWarnings.java} file.


The Daikon invariant detector
(\myurl{http://plse.cs.washington.edu/daikon/}) is also annotated with
\refqualclass{checker/interning/qual}{Interned}.  From directory \code{java},
run \code{make check-interning}.



\section{Other interning annotations\label{other-interning-annotations}}

The Checker Framework's interning annotations are similar to annotations used
elsewhere.

If your code is already annotated with a different interning
annotation, the Checker Framework can type-check your code.
It treats annotations from other tools
exactly as if you had written the corresponding annotation from the
Interning Checker, as described in Figure~\ref{fig-interning-refactoring}.


% These lists should be kept in sync with InterningAnnotatedTypeFactory.java .
\begin{figure}
\begin{center}
% The ~ around the text makes things look better in Hevea (and not terrible
% in LaTeX).
\begin{tabular}{ll}
\begin{tabular}{|l|}
\hline
 ~com.sun.istack.internal.Interned~ \\ \hline
\end{tabular}
&
$\Rightarrow$
~org.checkerframework.checker.interning.qual.Interned~
\end{tabular}
\end{center}
%BEGIN LATEX
\vspace{-1.5\baselineskip}
%END LATEX
\caption{Correspondence between other interning annotations and the
  Checker Framework's annotations.}
\label{fig-interning-refactoring}
\end{figure}



% LocalWords:  plugin MyInternedClass enum InterningExampleWithWarnings java
% LocalWords:  PolyInterned Alint dotequals quals InterningAnnotatedTypeFactory
% LocalWords:  javac InterningVisitor JLS Acheckclass UsesObjectEquals
%  LocalWords:  consing valueOf superclasses s2 cleanroom canonicalization
